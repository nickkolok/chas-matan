\section{Показательное распределение}

\subsection{Определение}
Говорят, что случайная величина $\xi$ имеет показательное распределение с параметром $\lambda>0$ и пишут: $\xi \sim \Pi(\lambda)$, если $\xi \sim \Gamma(1,\lambda)$.

\subsection{Матожидание и дисперсия}
Воспользовавшись определением, положим в формулах (\ref{matozhidanie_Gamma}) и (\ref{dispersia_Gamma}) $\nu=1$:

\begin{equation}
M\xi = 
\frac{1}{\lambda }
\end{equation}
\begin{equation}
D\xi = 
\frac{1}{\lambda^2 }
\end{equation}


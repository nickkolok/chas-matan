\section{Вырожденное распределение}

\subsection{Определение}
Говорят, что случайная величина $\xi$ имеет вырожденное распределение, если $\xi$ есть некоторая константа $c$ (т.е. случайная величина $\xi$ всегда принимает одно и то же значение).

\subsection{Матожидание}
Здесь и далее через $N_\xi$ обозначаем множество всех значений, которые может принимать дискретная случайная величина $\xi$.
В нашем случае $N_\xi = \{c\}$.

По формуле матожидания дискретной случайной величины:
\begin{equation}
M\xi =
\suml_{x_k \in N_\xi} x_k \cdot P\{\xi=x_k\} =
\suml_{x_k \in \{c\}} x_k \cdot P\{\xi=x_k\} =
c \cdot P\{\xi=c\} = 
c \cdot 1 = 
c
\end{equation}

\subsection{Дисперсия}
Сначала ищем матожидание квадрата:
\begin{equation}
M(\xi^2) =
\suml_{x_k \in N_\xi} x_k^2 \cdot P\{\xi=x_k\} =
\suml_{x_k \in \{c\}} x_k^2 \cdot P\{\xi=x_k\} =
\\=
c^2 \cdot P\{\xi=c\} = 
c^2 \cdot 1 = 
c^2
\end{equation}

По формуле дисперсии имеем:
\begin{equation}
D\xi = 
M(\xi^2)-(M\xi)^2 = 
c^2 - c^2 =
0
\end{equation}


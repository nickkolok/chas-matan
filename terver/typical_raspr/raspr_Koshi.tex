\section{Распределение Коши}

\subsection{Определение}
Говорят, что случайная величина $\xi$ имеет распределение Коши с параметрами $\mu\in\R$ и $\sigma>0$ и пишут: $\xi\sim C(\mu,\sigma)$, если
\begin{equation}
f_\xi (x) = \frac{1}{\pi\sigma}\cdot \frac{1}{1+ \left( \frac{x-\mu}{\sigma} \right)  ^2}
\end{equation}

\subsection{Отсутствие моментов (в т. ч. матожидания и дисперсии)}
Убедимся, что матожидания не сущеcтвует (а значит, не существует и моментов более высокого порядка, т. е. никаких моментов, в том числе дисперсии).
Напомним, что матожиданием абсолютно непрерывной случайной величины $\xi$ называется интеграл вида
\begin{equation}
 \intl_{-\infty }^{\infty } x f_\xi (x) dx
\end{equation}
в случае, если он сходится абсолютно.
Для распределения Коши имеем интеграл
\begin{equation}\label{matozhidanie_Koshi}
 \intl_{-\infty }^{\infty } x \frac{1}{\pi\sigma}\cdot \frac{1}{1+ \left( \frac{x-\mu}{\sigma} \right)  ^2} dx
\end{equation}
И этот интеграл по абсолютной величине расходится.
В самом деле,
\begin{multline*}
 \intl_{-\infty }^{\infty } \left|x \frac{1}{\pi\sigma}\cdot \frac{1}{1+ \left( \frac{x-\mu}{\sigma} \right)  ^2} \right| dx = 
 \\ \mbox{( положим $x=\sigma y$, тогда $dx=\sigma dy$ )} \\=
 \intl_{-\infty }^{\infty } \left| \frac{\sigma y}{\pi}\cdot \frac{1}{1+ \left( y-\frac{\mu}{\sigma} \right)  ^2} \right| dy = 
 \\ \mbox{( положим $\lambda = \frac{\mu}{\sigma}$ )} \\=
 \intl_{-\infty }^{\infty } \left| \frac{\sigma y}{\pi}\cdot \frac{1}{1+ \left( y-\lambda  \right)  ^2} \right| dy = 
 \\ \mbox{( положим $\alpha = 1+|\lambda|$ )} \\=
 \intl_{-\infty}^{\alpha} \left| \frac{\sigma y}{\pi}\cdot \frac{1}{1+(y-\lambda)^2} \right| dy + \intl_{\alpha}^{\infty } \left| \frac{\sigma y}{\pi}\cdot \frac{1}{1+( y-\lambda)^2} \right| dy \geq
 \\ \mbox{( т.к. подынтегральное выражение неотрицательно, можем оценить }\\ \mbox{ снизу сумму интегралов вторым интегралом )} \\ \geq
 \intl_{\alpha}^{\infty } \left| \frac{\sigma y}{\pi}\cdot \frac{1}{1+( y-\lambda)^2} \right| dy =
 \intl_{\alpha}^{\infty } \frac{\sigma |y|}{\pi}\cdot \frac{1}{1+( y-\lambda)^2} dy =
 \\ \mbox{( т.к. при $y\geq \alpha = 1+ |\lambda|>0$ имеем $|y|=y$ )} \\=
 \intl_{\alpha}^{\infty } \frac{\sigma y}{\pi}\cdot \frac{1}{1+( y-\lambda)^2} dy =
 \frac{\sigma}{\pi}\intl_{\alpha}^{\infty } \frac{y}{1+( y-\lambda)^2} dy =
 \\ \mbox{( положим $z=y-\lambda$, тогда $dz=dy$ )} \\=
 \frac{\sigma}{\pi}\intl_{\alpha}^{\infty } \frac{z+\lambda}{1+z^2} dz =
 \frac{\sigma}{\pi}\intl_{\alpha}^{\infty } \frac{z}{1+z^2} dz +  \frac{\sigma}{\pi}\intl_{\alpha}^{\infty } \frac{\lambda}{1+z^2} dz =
 \\=
 \frac{\sigma}{2\pi}\intl_{\alpha}^{\infty } \frac{2z}{1+z^2} dz +  \frac{\sigma\lambda}{\pi}\intl_{\alpha}^{\infty } \frac{1}{1+z^2} dz =
 \\=
 \frac{\sigma}{2\pi}\intl_{\alpha}^{\infty } \frac{2z}{1+z^2} dz +  \frac{\sigma\lambda}{\pi}\left. (  \arctg z)  \right|_{ z=\alpha}^{z=+\infty } =
 \\ 
\end{multline*}
\begin{multline}
 =
 \frac{\sigma}{2\pi}\intl_{\alpha}^{\infty } \frac{2z}{1+z^2} dz +  \frac{\sigma\lambda}{\pi}(  \arctg (+\infty) - \arctg \alpha) =
 \\=
 \frac{\sigma}{2\pi}\intl_{\alpha}^{\infty } \frac{2z}{1+z^2} dz +  \frac{\mu}{\pi}( \frac{\pi}{2} - \arctg \alpha) =
 \\ \mbox{( к интегралу справа прибавляется конечная константа, зависящая  }\\ \mbox{ от параметров распределения, обозначим её через $\beta$ )} \\=
 \frac{\sigma}{2\pi}\intl_{\alpha}^{\infty } \frac{2z}{1+z^2} dz + \beta =
 \\ \mbox{( положим $t=1+z^2$, тогда $dt=2zdz$ )} \\=
 \frac{\sigma}{2\pi}\intl_{\alpha}^{\infty} \frac{1}{t} dt + \beta =
 \frac{\sigma}{2\pi}\left. \left( \ln |t| \right)  \right|_{t=\alpha }^{t=+\infty } + \beta =
 \\=
 \frac{\sigma}{2\pi} \left( \ln \left|+\infty\right| - \ln |\alpha| \right) + \beta =
 \frac{\sigma}{2\pi} \left( +\infty - \ln \left(1+\left|\frac{\mu}{\sigma}\right|\right) \right) + \beta = +\infty
\end{multline}

Следовательно, интеграл (\ref{matozhidanie_Koshi}) сходится не абсолютно, и ни математического ожидания, ни каких-либо других моментов (включая дисперсию) у распределения Коши не существует.


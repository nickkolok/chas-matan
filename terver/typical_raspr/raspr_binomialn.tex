\section{Биномиальное распределение}

\subsection{Определение}
\begin{equation}
P\{\xi=k\} =  C^{ k}_{ n} p^k (1-p)^{n-k}
\end{equation}

\subsection{Матожидание}

И снова заметим, что сумма вероятностей равна 1:
\begin{equation}\label{summa_veroyatn_binomialn}
\sum_{ l=0}^{m} C^{l}_{m} p^{l} (1-p)^{m-l} = 1
\end{equation}

\begin{multline}
M\xi = 
 \sum_{ k=0}^{ n} k C^{ k}_{ n} p^k (1-p)^{n-k} =
 \\ \mbox{( при $k=0$ под суммой всё равно $0$ )} \\=
 \sum_{ k=1}^{ n} k C^{ k}_{ n} p^k (1-p)^{n-k} \stackrel{(\ref{ponizhenie_C})}{=}
 \sum_{ k=1}^{ n} k \frac{n}{k}C^{ k-1}_{ n-1} p^k (1-p)^{(n-1)-(k-1)} =
 \\=
 \sum_{ k=1}^{ n} n C^{ k-1}_{ n-1} p^{k-1} p (1-p)^{(n-1)-(k-1)} =
 np \sum_{ k=1}^{ n} C^{ k-1}_{ n-1} p^{k-1} (1-p)^{(n-1)-(k-1)} =
 \\ \mbox{( положив $l=k-1$, $m=n-1$ )} \\=
 np \sum_{ l=0}^{m} C^{l}_{m} p^{l} (1-p)^{m-l} \stackrel{(\ref{summa_veroyatn_binomialn})}{=}
 np
\end{multline}

\subsection{Дисперсия}
\begin{multline*}
 D\xi = 
 M(xi^2)-(M\xi)^2 = 
 \sum_{ k=0}^{ n} k^2 C^{ k}_{ n} p^k (1-p)^{n-k} - n^2p^2 =
 \\ \mbox{( при $k=0$ под суммой всё равно $0$ )} - n^2p^2 \\=
 \sum_{ k=1}^{ n} k^2 C^{ k}_{ n} p^k (1-p)^{n-k} - n^2p^2 \stackrel{(\ref{ponizhenie_C})}{=}
 \sum_{ k=1}^{ n} k^2 \frac{n}{k}C^{ k-1}_{ n-1} p^k (1-p)^{(n-1)-(k-1)} - n^2p^2 =
 \\=
 \sum_{ k=1}^{ n} k nC^{ k-1}_{ n-1} p^k (1-p)^{(n-1)-(k-1)} - n^2p^2 =
 \\=
 \sum_{ k=1}^{ n} (k-1+1) nC^{ k-1}_{ n-1} p^k (1-p)^{(n-1)-(k-1)} - n^2p^2 =
 \\=
 \sum_{ k=1}^{ n} (k-1) nC^{ k-1}_{ n-1} p^k (1-p)^{(n-1)-(k-1)} +\sum_{ k=1}^{ n} nC^{ k-1}_{ n-1} p^k (1-p)^{(n-1)-(k-1)} - n^2p^2 =
 \\
\end{multline*}
\begin{multline}
 =
 n\sum_{ k=1}^{ n} (k-1) C^{ k-1}_{ n-1} p^k (1-p)^{(n-1)-(k-1)} +\\+ n \sum_{ k=1}^{ n} C^{ k-1}_{ n-1} p^k (1-p)^{(n-1)-(k-1)} - n^2p^2 =
 \\ \mbox{( положив $l=k-1$, $m=n-1$ )} \\=
 n\sum_{ l=0}^{ m} l C^{l}_{m} p^l \cdot p (1-p)^{m-l} + n \sum_{ l=0}^{ m} C^{l}_{m} p^l \cdot p (1-p)^{m-l} - n^2p^2 =
 \\=
 np\sum_{ l=0}^{ m} l C^{l}_{m} p^l (1-p)^{m-l} + np \sum_{ l=0}^{ m} C^{l}_{m} p^l (1-p)^{m-l} - n^2p^2 =
 \\ \mbox{( но правая сумма соответствует (\ref{summa_veroyatn_binomialn}) )} \\=
 np\sum_{ l=0}^{ m} l C^{l}_{m} p^l (1-p)^{m-l} + np - n^2p^2 =
 \\ \mbox{( при $l=0$ под суммой всё равно $0$ )} \\=
 np\sum_{ l=1}^{ m} l C^{l}_{m} p^l (1-p)^{m-l} + np - n^2p^2 \stackrel{(\ref{ponizhenie_C})}{=}
 \\=
 np\sum_{ l=1}^{ m} l \frac{m}{l} C^{l-1}_{m-1} p^l (1-p)^{m-l} + np - n^2p^2 =
 \\=
 nmp\sum_{ l=1}^{ m} C^{l-1}_{m-1} p^{l-1} \cdot p (1-p)^{(m-1)-(l-1)} + np - n^2p^2 =
 \\=
 n(n-1)p^2\sum_{ l=1}^{ m} C^{l-1}_{m-1} p^{l-1} (1-p)^{(m-1)-(l-1)} + np - n^2p^2 =
 \\ \mbox{( но такую сумму мы уже считали пи вычислении матожидания )} \\=
 n(n-1)p^2 + np - n^2p^2 =
 n^2 p^2 - np^2+ np - n^2 p^2 =
 - np^2+ np =
 np(1-p)
\end{multline}


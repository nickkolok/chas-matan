\documentclass[a4paper,14pt]{article}
\usepackage[T2A]{fontenc}
\usepackage[utf8]{inputenc} % любая желаемая кодировка
\usepackage[russian,english]{babel}
\usepackage[pdftex,unicode]{hyperref}
\usepackage{indentfirst} % включить отступ у первого абзаца
\usepackage{amssymb}

\begin{document} % начало документа
\large

Лемма 1 (доказана в теме "Применение фокального свойства гиперболы")

Если в системе целочисленно удалённых точек (далее СЦТ) найдутся три точки
$M_1$, $M_2$ и $M_3$, не лежащие на одной прямой, и 
$a=|M_1 M_2| \in \mathbb{N}$,
$b=|M_1 M_3| \in \mathbb{N}$,
$c=|M_2 M_3| \in \mathbb{N}$,
то максимальное количество точек в такой СЦТ равно 
$4\cdot\min\{ab,ac,bc\}$.

Лемма 2

Если в СЦТ найдётся $\beta = 2m^2 +1$ точек, никакие три из которых не лежат на одной прямой, и эта СЦТ лежит в пределах квадрата со стороной $n$, то $n > \frac{\beta - 1}{4}$ (иначе говоря, $ \beta < 4n +1$).

Доказательство

СЦТ лежит в пределах квадрата со стороной $n$. Разобьём этот квадрат на $m^2$ меньших равных между собой квадратов со стороной $\frac{n}{m}$. Тогда по принципу Дирихле найдётся хотя бы один квадрат со стороной $\frac{n}{m}$, внутри которого (возможно, включая границы) найдутся три точки, принадлежащие рассматриваемой СЦТ. Обозначим их через $M_1$, $M_2$ и $M_3$. Ни одно из расстояний $|M_1 M_2|$, $|M_1 M_3|$ и $|M_2 M_3|$, очевидно, не превышает диагонали квадрата со стороной $\frac{n}{m}$, т. е. $\frac{n}{m}\sqrt{2}$. Тогда по лемме 1 количество точек в СЦТ $\beta \le \frac{8n^2}{m^2}$. Имеем:
$$ 2m^2+1 \le \frac{8n^2}{m^2}$$
$$ 2m^2 < 2m^2+1 \le \frac{8n^2}{m^2}$$
$$ 2m^2 < \frac{8n^2}{m^2}$$
$$ m^2 < \frac{4n^2}{m^2}$$
$$ m^4 < 4n^2$$
Т. к. $n$ положительно, извлекаем корень:
$$ m^2 < 2n$$
$$ 2m^2 +1 < 4n +1$$
$$ \beta < 4n +1$$
$$n > \frac{\beta - 1}{4}$$
Лемма доказана.

Утверждение 1 (вспомогательное)

$\forall \left(\beta \in \mathbb N\right)\left[  2 \sqrt{\frac{\beta - 1}{2}} < \beta \right]$

Доказательство

Т. к. $\beta >0$, возводим обе части неравенства в квадрат:

$$4 \frac{\beta - 1}{2} < \beta^2$$
$$ \beta^2-2\beta+2>0$$
$$ \beta^2-2\beta+1>-1$$
$$ (\beta-1)^2>-1$$

Утверждение доказано.

Лемма 3.

Пусть СЦТ состоит из $\gamma$ точек, никакие три из которых не лежат на одной прямой, и лежит внутри квадрата со стороной $n$. Тогда $\gamma<12n+4$.

Доказательство

Возьмём $m \in \mathbb{N}$ такое, что $2m^2+1 \le \gamma \le 2(m+1)^2$ (это можно сделать единственным образом).
Обозначим $2m^2+1=\beta$, откуда $m=\sqrt{\frac{\beta-1}{2}}$. Тогда по лемме 2 имеем $ \beta < 4n +1$. Оценим $\gamma$:

$$ \gamma \le 2(m+1)^2 = 2m^2+4m +2 < \beta + 1 + 2 \cdot 2 \sqrt{\frac{\beta-1}{2}} < 
\beta + 1 + 2 \beta = 3\beta+1<12n+4$$

Лемма доказана.



\end{document} % конец документа

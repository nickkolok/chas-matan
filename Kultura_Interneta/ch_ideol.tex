Идеологические течения в интернете весьма разнообразны; мы рассмотрим те из них, которые специфичны именно для высокотехнологичной среды общения.

\section{Анонимус.}

Основная идея анонимуса -- отказ от имени, отказ от персонификации. Это течение зародилось в начале 2000-х на имиджбордах -- специализированных веб-сайтах, на которых отсутствует регистрация и возможность как-либо идентифицировать пользователя. Основной структурной единицей имиджборды является тред - тема, в которой анонимусы оставляют посты - сообщения, к котором можно прикреплять изображения (отсюда и название).

Позже возникла хакерская группировка Anonymous.

Анонимусы характеризуются агрессивностью, пристрастием к обсценной лексике -- с одной стороны и жаждой конструктивной деятельности и всеобщей справедливости -- с другой.


\section{Пиратство и OpenSource.}

Эти два течения объединяет стремление ограничить монополию на копирование, называемую ныне авторским правом. Фактически сейчас сложилась ситуация, при которой совершенно непричастные к созданию произведению люди (например, наследники или т. наз. ``правообладатели'') получают выгоду от ограничений, накладываемых на использование этого произведения.

Отдельной строкой следует отметить здесь варварскую политику научных журналов, которые берут плату как с учёных, публикующих статьи, так и с библиотек, эти журналы приобретающих, при всём этом ещё и злоупотребляя своим монопольным положением. В общеобразовательной школе ситуация аналогична: зачастую просто не существует легальных электронных версий учебников.

Программное обеспечение во многом контролируется фирмами-гигантами вроде apple и microsoft, получившими контроль над всеми данными пользователей.

Всё вышеприведённое дало толчок развитию движений, направленных на ограничение копирайта.

Наименее радикальная ветвь - движение OpenSource, призывающее авторов (и в первую очередь программистов) добровольно отказаться от права ограничивать использование произведения. Именно благодаря OpenSource-программистам существует, например, операционная система Android.

Умеренные пираты считают, что некоммерческое использование произведения должно быть разрешено независимо от воли автора. Существуют и радикальные пираты, ратующие за полную отмену авторских прав.
 

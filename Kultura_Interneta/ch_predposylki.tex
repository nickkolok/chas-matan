Обособление субкультуры интернета от программистского и инженерного сообщества началось тогда, когда интернет (здесь и далее это слово пишем с маленькой буквы, подобно тому, как с маленькой буквы пишутся атмосфера и небо) перестал быть исключительно средством передачи технической, формальной информации и превратился в средство общения, а затем и развлечения; впрочем, развлекательный сегмент Всемирной сети мы в данной работе почти не затронем.

Основных предпосылок обособления субкультуры интернета несколько. Рассмотрим наиболее важные из них.

\paragraph{Высокая скорость и точность копирования и модификации информации.}

Чтобы переписать стихотворение -- нужно несколько минут. Чтобы издать книгу -- нужно несколько месяцев. Чтобы нажать заветные Ctrl+C и Ctrl+V (копировать и затем вставить, на луркоязе (см. далее) -- ``скопипастить'') --нужно несколько секунд. Информация распространяется в интернете практически мгновенно и с дискретной точностью (без малых искажений, какие возникают, например, при ксерокопировании). Интернет позволяет передавать данные нескольких типов (текст, звук, изображение, видео, документы и т. д.) через единый интерфейс.

\paragraph{Низкая персонифицируемость.}

Связать аккаунт (учётную запись, профиль) на интернет-ресурсе или даже отдельное сообщение с реальным человеком, если он сам того не желает, другому рядовому пользователю не под силу. Это позволяет, например, создать видимость массовой поддержки мнения, или вести две разных ``виртуальных жизни'', и тому подобное.

\paragraph{Связность и адресуемость.}

Любой компьютер и вообще ЭВМ (например, современный мобильный телефон) может обратиться к любой другой ЭВМ, если знает её адрес. Каждая подключенная к интернету ЭВМ имеет адрес, но, как правило, у пользовательских ЭВМ этот адрес непостоянен (из-за исчерпаемости пространства IPv4), в то время как у серверных ЭВМ, т. е. ЭВМ, отвечающих за работоспособность сайтов, адрес постоянен.

\paragraph{Первоначальная элитарность}

На заре интернета в современном его виде существовал некий естественный ценз -- как имущественный, так и интеллектуальный -- который позволял "интеретчикам" чувствовать себя закрытым и элитарным сообществом. Впоследствии ценз сильно упал (хотя ещё де-факто остаётся), но вот желание быть элитой никуда не пропало. Такова уж, видно, человеческая природа...





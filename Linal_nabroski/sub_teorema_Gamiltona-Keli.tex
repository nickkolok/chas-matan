\begin{teorema}[(Гамильтона-Кэли)]
Характеристический многочлен $\varphi(\lambda)$ линейного оператора $A:R^n\to R^n$ является аннулирующим многочленом оператора $A$.
\end{teorema}

\dokvo
Пусть $e$ -- какой-нибудь базис $R^n$ и $A_e$ - матрица оператора $A$ в нём.
Обозначим $B=A_e-\lambda I$, $\tilde{B}$ -- матрицу, присоединённую к $B$, т.~е.
$$
\tilde{B}=\begin{pmatrix}
|B_{11}| & \hdots & |B_{n1}| \\
\hdotsfor{3}\\
|B_{1n}| & \hdots & |B_{nn}| 
\end{pmatrix}
$$

Мы знаем, что $\tilde{B} B = |B|I$, откуда
$$
\tilde{B}(A_e-\lambda I)=\varphi(\lambda) I
$$

С другой стороны, элементы матрицы $\tilde{B}$ являются многочленами от $\lambda$, притом их степень не превосходит $n-1$. Значит,
$$
\tilde{B}=B_0\lambda^{n-1}+B_1\lambda^{n-2}+...+B_{n-1}
$$

Пусть многочлен $\varphi(\lambda)$ имеет вид
$$
\alpha_0 \lambda^n+\alpha_1 \lambda^{n-1}+...+\alpha_n
$$

Тогда

$$
(B_0\lambda^{n-1}+B_1\lambda^{n-2}+...+B_{n-1})(A_e-\lambda I)=(\alpha_0 \lambda^n+\alpha_1 \lambda^{n-1}+...+\alpha_n)I
$$

Рассмотрим это выражение как равенство многочленов относительно $\lambda$. Раскрыв скобки, выпишем коэффициенты при одинаковых степенях $\lambda$ и приравняем их:
$$~~-B_0=\alpha_0 I$$
$$B_0 A_e-B_1 = \alpha_1 I$$
$$B_1 A_e-B_2 = \alpha_2 I$$
$$.........$$
$$B_{n-2} A_e-B_{n-1} = \alpha_{n-1} I$$
$$B_{n-1} A_e~~ = \alpha_{n} I$$

Умножая эти равенства справа соответственно на $A_e^{n}$,$A_e^{n-1}$, ..., $A_e$, $I$ и складывая, получим
$$
0=\alpha_0 A_e^{n}+\alpha_1 A_e^{n-1}+\alpha_n I
$$

Таким образом, $\varphi(A_e)=0$, следовательно, оператор $\varphi(\alpha)$ -- нулевой



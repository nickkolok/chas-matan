\begin{opr}
	Число $b$ называется пределом последовательности $\{x_n\}$, если
	\begin{equation}
		\forall(\varepsilon > 0)\exists(N \in \mathbb{N})\forall(n>N)[|x_n-b|<\varepsilon]
		.
	\end{equation}
	Пишут:
	\begin{equation}
		\lim_{x\to \infty} x_n = b
		.
	\end{equation}
\end{opr}

Таким образом, предел "--- это некий <<идеал>>\footnote{
Это слово взято в кавычки, поскольку в алгебре используется как термин и несёт совсем другой смысл
},
к которому последовательность стремится.
Не всегда она к нему стремится неуклонно, может иногда <<срываться>>,
откатываться <<назад>>,
но сколь бы малый <<зазор>> от предела мы не взяли,
всегда найдётся член последовательности,
начиная с которого все члены последовательности попадают в этот <<зазор>>.


Вообще говоря, предела у последовательности может и не существовать.

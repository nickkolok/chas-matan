\subsubsection{Теорема Лагранжа о промежуточном значении (о конечных приращениях).}

Если $f:[a;b]\to \R$ такова, что

1) $f$ непрерывна на $[a;b]$;

2) $f$ дифференцируема на $(a;b)$;

то $\exists(c \in (a;b))[f(b)-f(a)=f'(c)(b-a)]$.

\subsubsection{Замечание 1.}

Равенство $f(b)-f(a)=f'(c)(b-a)$ называют формулой Лагранжа или формулой конечных приращений.

\subsubsection{Замечание 2.}

Формулу Лагранжа можно записать и в другом виде, если положить $\theta=\frac{c-a}{b-a}$:

$$
f(b)-f(a)=f'(a+\theta(b-a))(b-a)
$$

Полагая $x=a, h=b-a$, имеем

$$
f(x+h)-f(x)=f'(x+\theta h)h
$$

\subsubsection{Следствие 1.}

Функция, имеющая на промежутке равную нулю производную, постоянная на нём.

\subsubsection{Следствие 2.}

Пусть на промежутке $X$ определены и дифференцируемы две функции $f$ и $g$, притом на концах промежутка, если они в него входят, $f$ и $g$ непрерывны.
Если $\forall(x \in X)[f'(x)=g'(x)]$, то $\forall(x \in X)[f(x)-g(x)=const]$.

\subsubsection{Следствие 3.}

Функция, имеющая на промежутке ограниченную производную, равномерно непрерывна на нём.

\subsubsection{Следствие 4.}

Пусть $f:[a;b]\to \R$, $f$ непрерывна, $f$ дифференцируема на $(x_0;x_0+h)\subset [a;b]$.
Тогда правая производная $f$ в $x_0$ непрерывна.



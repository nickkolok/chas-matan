\begin{opred}
Пусть есть $f:[a;b] \to R$, f(x) - ограничена.
Есть $\forall(T:T=\{x_0,...,x_n\}); M_i=\sup_{x\in\vartriangle_i}f(x); m_i=\inf_{x\in\vartriangle_i}f(x).$
\\
Тогда верхняя ($\overline{S}(f,T)$)/нижняя ($\underline{S}(f,T)$) сумма Дарбу функции f(x) по отрезку [a;b] и разбиению T - это:
$$
\overline{S}(f,T)=\sum_{i=1}^{n}M_i \vartriangle x_i
$$
$$
\underline{S}(f,T)=\sum_{i=1}^{n}m_i \vartriangle x_i
$$
где $\vartriangle x_i = x_i - x_{i-1}; \vartriangle_i=[x_{i-1};x_i]$
\end{opred}

\subsubsection{Свойства сумм Дарбу}
\subsubsection{(1)}
$\forall(T)$ справедливо неравенство:
$$
m(b-a) \leq \underline{S}(f,T)\leq S(f,T)\leq \overline{S}(f,T) \leq M(b-a)
$$
$$ где m=\inf_{x \in [a;b]}f(x), M=\sup_{x\in[a;b]}f(x)$$
$$S(f,(T,\varphi))=\sum_{i=1}^{n}f(\varphi_i)\vartriangle x_i  (\varphi_i \in \vartriangle_i)$$
$$m\leq m_i\leq f(\varphi_i)\leq M_i\leq M$$
\subsubsection{(2)}
$\forall(T)[\sup S(f,(T,\varphi))=\overline{S}(f,T);\inf S(f,(T,\varphi))=\underline{S}(f,T)]$
\\
\dokvo
Пусть $\varphi=(\varphi_1,...,\varphi_n)$
\\
$\sup S(f,(T,\varphi))=\sup \sum_{i=1}^{n}f(\varphi_i)\vartriangle x_i=\sup \sum_{i=1}^{n}\sup f(\varphi_i)\vartriangle x_i=$
\\
$=\sum_{i=1}^{n}M_i\vartriangle x_i = \overline{S}(f,T)$
\\
$\inf S(f,(T,\varphi))=\inf\sum_{i=1}^{n}(f(\varphi_i)\vartriangle x_i)=\inf\sum_{i=1}^{n}\inf f(\varphi_i)\vartriangle x_i = \sum_{i=1}^{n}m_i\vartriangle x_i = \underline{S}(f,T)$
\subsubsection{(3)}
$\forall(T,\tilde{T}:T\subset\tilde{T})[\overline{S}(f,\tilde{T})\leq \overline{S}(f,T); \underline{S}(f,\tilde{T})\geq \underline{S}(f,T)]$
\dokvo
$$\overline{S}(f,\tilde{T})=\sum_{i=1}^{n}\sum_{j=1}^{n_i}M_{ij}\vartriangle x_{ij} \leq \sum_{i=1}^{n}M_i\sum_{j=1}^{n_i}\vartriangle x_{ij}= \sum_{i=1}^{n}M_i\vartriangle x_i = \overline{S}(f,T)
$$
$$
M_{ij}=\sup_{x\in \vartriangle_{ij}}
$$

$$\underline{S}(f,\tilde{T})=\sum_{i=1}^{n}\sum_{j=1}^{n_i}m_{ij}\vartriangle x_{ij} \geq \sum_{i=1}^{n}m_i\sum_{j=1}^{n_i}\vartriangle x_{ij}= \sum_{i=1}^{n}m_i\vartriangle x_i = \underline{S}(f,T)
$$
$$
M_{ij}=\inf_{x\in \vartriangle_{ij}}
$$
\dokno

\subsubsection{(4)}
Пусть $\forall(T_1,T_2)\exists(T)[T=T_1 \cup T_2].$ Тогда $T_1 \subset T, T_2 \subset T.$
$$\underline{S}(f,T_1) \leq \underline{S}(f,T) \leq \overline{S}(f,T) \leq \overline{S}(f,T_2)$$
\dokno

\subsubsection{Верхние и нижние интегралы Дарбу}
Из свойства и сумм Дарбу следует, что ${\overline{S}(f,T)}$ ограничена снизу, т.е. $\exists(\inf_{T}\overline{S}(f,T))$ и $\underline{S}(f,T)$ ограничена сверху, т.е. $\exists(\sup_{T}(f,T))$
\\
\begin{opred}
Эти числа называются соответственно верхний($\overline{\overline{s}}$) и нижний ($\underline{\overline{s}}$) интегралы Дарбу.
$$
\overline{\overline{s}}=\inf_{T}\overline{S}(f,T);\underline{\overline{s}}=\inf_{T}\underline{S}(f,T)
$$
\end{opred}

\begin{teorema}
$$
\lim_{d(T)\to 0}\overline{S}(f,T)=\overline{\overline{s}};
\lim_{d(T)\to 0}\underline{S}(f,T)=\underline{\overline{s}}
$$
\end{teorema}
\dokvo
Надо доказать что:
$$
\forall(\epsilon>0)\exists(\sigma>0)\forall(T:d(T)<\sigma)[|\overline{S}(f,T)-\overline{s}|<\epsilon].
$$
$$
\overline{\overline{s}}=\inf\overline{S}(f,T)\Rightarrow\overline{S}(f,T)-\overline{\overline{s}}\geq 0\Rightarrow|\overline{S}(f,T)-\overline{s}|=\overline{S}(f,T)-
\overline{s}
$$
\\
Значит теперь надо доказать:
$$
\forall(\epsilon>0)\exists(\sigma>0)\forall(T:d(T)<\sigma)[\overline{S}(f,T)<\overline{s}+\epsilon]
$$
Возьмем $\forall(\epsilon>0)$ т.к. $\overline{s}=\inf \overline{S}(f,T)$, то по определению $\inf:$
$$\exists(T_1 отрезка [a;b]=\{x_0,...,x_n\})[\overline{S}(f,T)<\overline{\overline{s}}+\frac{\epsilon}{2}]$$
Положим $\exists(\sigma=\frac{\epsilon}{2\Omega p})$, где $\Omega=\omega(f,[a;b])$ p - количество точек в $T_1.$
\\
Возьмем произвольное разбиение $\forall(T:d(T)<\sigma).$
Пусть $T_2=T \cup T_1;$ тогда $T \subset T_2, T_1 \subset T.$
\\
1). Если $\exists(\vartriangle_i \subset T)[$в $\vartriangle_i$ нет $x_i].$ Тогда $M_i \vartriangle x_i = 0.$
\\
2).Если в $\vartriangle_i$ из T есть точки разбиения $T_1(n_j)$, тогда посчитаем разность:
$$
\overline{S}(f,T)-\overline{S}(f,T_2)=M_i\vartriangle x_i - \sum_{j=1}^{n_i}M_{ij}\vartriangle x_{ij} = \sum_{j=1}^{n_i}(M_i - M_{ij})\vartriangle x_{ij} \leq 
$$
$$
\leq (M_i - M_{ij}=\sup_{[a;b]}f(x)-\sup_{[x_i;x_{i+1}]_ni}f(x) \leq\omega(f,[a;b])=\Omega) \leq
$$
$$
\leq\Omega\sum_{j=1}^{n}\vartriangle x_{ij} =\Omega\vartriangle x_i<\Omega d(T)<\Omega\sigma=\frac{\epsilon}{2p}
$$
На каждом из подотрезков, рассматриваемого типа, внутри которых есть точки из $T_1$, меньше p.
\\
Поэтому
$$
\overline{S}(f,T)-\overline{S}(f,T_2)<p\cdot\frac{\epsilon}{2p}=\frac{\epsilon}{2}
$$
Т.к. $T_2$ - измельчение разбиения $T_1$, то по свойству №3 верхних сумм Дарбу:
$$
\overline{S}(f,T_2)\leq\overline{S}(f,T_1)<\overline{s}+\frac{\epsilon}{2}
$$
Отсюда: $\overline{S}(f,T_1)-\overline{s}<\frac{\epsilon}{2}.$
$$\overline{S}(f,T)<\frac{\epsilon}{2}+\overline{S}(f,T_2)<\frac{\epsilon}{2}+\overline{s}+\frac{\epsilon}{2}<\overline{s}+\epsilon$$
\dokno




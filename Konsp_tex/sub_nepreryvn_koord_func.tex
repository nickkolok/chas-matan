Определение непрерывности для случая векторной функции векторного аргумента мы не будем давать таким же образом, как давали определения для скалярных функций скалярного или векторного аргумента.
Вместо этого дадим сначала определение через непрерывность координатных функций, а затем докажем эквивалентность ему классических определений по Коши и по Гейне.

\begin{opr}
Если $f:A\subset\R^n\to\R^m$ и $\forall(x\in A)[f(x)=(f^1(x), ... , f^m(x))]$, то скалярные функции векторного аргумента $f^i:A\to\R$ называются координатными функциями исходной функции $f$.
\end{opr}

\begin{opr}\label{opred_predela_Rn_Rm_nepr}
Если $f:A\subset\R^n\to\R^m$ и $\forall(x\in A)[f(x)=(f^1(x), ... , f^m(x))]$, то функция $f$ называется непрерывной в точке $x_0\in A$, если в этой точке непрерывны все её координатные функции.
\end{opr}

\begin{opr}
Если функция $f$ непрерывна в каждой точке множества $A$, то она непрерывна на это множестве.
\end{opr}

Заметим, что из данного таким образом определения следует непрерывность функции в изолированной точке её области определения.


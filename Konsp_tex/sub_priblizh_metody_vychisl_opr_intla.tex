В технических задачах вычислять определённый интеграл по формуле Ньютона-Лейбница или, тем более, по определению часто бывает очень сложно и не нужно: требуется только определённая точность.
В этих случаях подынтегральную функцию заменяют функцией более простой, как правило, кусочно-непрерывной.

Пусть $f$ -- функция, которую надо численно проинтегрировать на $[a;b]$, через $g$ обозначим функцию, которой будем заменять $f$.

Метод первый -- метод прямоуголника -- фактически повторяет определение, но "идёт не до конца":
строится разбиение $T$ с точками деления $x_0, ..., x_n$; эти же точки принимаются за отмеченные на отрезках, левыми (правыми) концами которых они являются (одна точка, конечно, остаётся лишней).
Таким образом, при методе прямоугольника функция $g$ принимает вид (в случае правых концов):
$$
g(x)=f(x_i), \mbox{~где~} x_{i-1}<x\leq x_i
$$

Метод трапеции предполагает замену функции на ломаную с вершинами $(x_i,f(x_i))$.

Однако на практике наиболее часто используется метод Симпсона, или метод парабол.
Он основан на том, что неизвестные коэффициенты функции $g_i(x)=a_i x^2 + b_i x + c_i$ можно восстановить по трём точкам, принадлежащим грайику этой функции.
Отрезок $[a;b]$ разбивают на $n=2m$ частей, а затем на отрезках $[x_{2i};x_{2i+2}]$ заменяют параболами, проходящими через точки $(x_{2i},f(x_{2i})$, $(x_{2i+1},f(x_{2i+1})$, $(x_{2i+2},f(x_{2i+2})$.




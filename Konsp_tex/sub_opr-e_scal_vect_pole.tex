\begin{opr}
Скалярным полем называется функция, действующая из \Rn в $\R$.
\end{opr}

\begin{opr}
Векторным полем называется функция, действующая из \Rn в \Rm.
\end{opr}

В рамках нашего курса будем работать, если не оговорено обратное, со случаем $n=m=3$.

Здесь у пытливого читателя естественным образом возникает недоумённый вопрос: зачем нужны новые термины, когда есть уже знакомые и привычные ``скалярная функция векторного аргумента'' и ``векторная функция векторного аргумента''? Дело даже не столько в том, что ``поле'' звучит короче и удобнее, чем ``функция векторного аргумента'', сколько в том, что поля используются в основном в приложениях, где первично то, что поле изначально задано в каждой точке, а система координат вводится уже после; с этой особенностью полей мы ещё столкнёмся, когда будем говорить об инвариантности некоторых характеристик поля относительно ортогональных преобразований системы координат.

Введём теперь несколько характеристик векторного поля. Пусть в области $G$ задано векторное поле $\vec{a}=\vec{a}(M)$.

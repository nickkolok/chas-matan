\section{Ряды Фурье}
\subsection{Линейные пространства со скалярным произведением. Ортогональные и ортонормированные системы}
\subsection{Теорема о существовании и единственности проекции на $X_n$. Неравенство Бесселя}
\begin{teorema}
Пусть $X_n=\{x_k\}$ --- произвольная ортонормированная система в линейном пространстве $X$.
Тогда $\forall(y\in X)\forall(n\in\N)$ существует единственная проекция $x_0^n$ элемента $y$ на подпространство $X_n$.
\end{teorema}

%TODO: доказательство

\begin{sledstvie}[Неравенство Бесселя]
$$
\forall(y\in X)\left[ \sum_{k=1}^\infty \left< y; x_k \right>^2 \leq \|y\|^2 \right]
$$
\end{sledstvie}

\begin{sledstvie}
$$
\left\|y-\sum_{k=1}^n \left<y; x_k\right> x_k \right\|^2 = \left\|y\right\|^2-\sum_{k=1}^n \left<y; x_k\right>^2
$$
\end{sledstvie}


\subsection{Ряд Фурье по произвольной ортонормированной последовательности элементов. Минимальное свойство частичных сумм ряда Фурье}
\subsection{Критерий замкнутости ортонормированной системы}
\subsection{Полные ортонормированные системы. Связь между замкнутостью и полнотой}
\subsection{Тригонометрическая система функций, её ортонормированность. Тригонометрический ряд Фурье. Минимальное свойство. Неравенство Бесселя}
\subsection{Лемма Римана}
\begin{lemma}[Римана]
Пусть $f\in R[a;b]$.
Тогда при $\alpha \to\infty$
$$
\intl_a^b f(x) \cos \alpha x dx \to 0
$$
$$
\intl_a^b f(x) \sin \alpha x dx \to 0
$$
\end{lemma}

\subsection{Интегральное представление для частичных сумм ряда Фурье. Интеграл Дирихле}
\begin{utverzhd}
Частичная сумма ТРФ выражается формулой
$$
S_{2n+1} = \frac{1}{2\pi}\intl_0^\pi\left(f(x+\tau)+f(x-\tau)\right)\frac{\sin\left(n+\frac{1}{2}\right)\tau}{\sin\frac{\tau}{2}} d\tau
$$
\end{utverzhd}

В частности, для $f(x) = 1$ имеем

$$
1=\frac{1}{\pi}\intl_0^\pi\frac{\sin\left(n+\frac{1}{2}\right)\tau}{\sin\frac{\tau}{2}}d\tau
$$

\subsection{Принцип локализации Римана}
\subsection{Теорема о поточечной сходимости тригонометрического ряда Фурье}
\subsection{Теорема о равномерной сходимости тригонометрического ряда Фурье}
\subsection{Почленное дифференцирование и интегрирование рядов Фурье}
\subsection{Ряд Фурье для функции, определённой и интегрируемой на $[-l;l]$}
\subsection{Ряды Фурье для чётных и нечётных функций}
\subsection{Разложение в ряд Фурье функции, заданной на $[0;l]$}
\subsection{Комплексная форма ряда Фурье}
\subsection{Понятие о ряде Фурье для функции нескольких переменных}
\subsection{Понятие об интеграле Фурье}
\subsection{Понятие о преобразовании Фурье и его применении}
\subsection{Замкнутость в $L_2^1 [-\pi;\pi]$ системы тригонометрических функций (теорема Дирихле-Ляпунова) }
...

\section{Равномерная аппроксимация функций}
\subsection{Теорема Вейерштрасса об аппроксимации непрерывных функций с помощью тригонометрических многочленов}
\subsection{Теорема Вейерштрасса об аппроксимации непрерывных функций с помощью алгебраических многочленов}
...


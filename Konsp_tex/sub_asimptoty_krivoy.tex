Пусть $L$ -- кривая, заданная уравнением $y=f(x)$, $x \in X$, $y \in Y$.

\opred

Кривая $L$ имеет бесконечные ветви, если по крайней мере одно из множеств $X$ или $Y$ является неограниченным.

Рассмотрим функцию $\rho(x)=\sqrt{x^2+f^2(x)}$, $x \in X$. Для того, чтобы кривая $L$ имела бесконечные ветви, необходимо и достаточно, чтобы $\rho$ была неограниченна на $X$.

\opred

Прямая $x=x_0$ называется вертикальной асимптотой кривой $L$, заданной уравнением $y=f(x)$, если $f(x) \to \pm \infty$ при $x \to x_0 \pm$, т. е. один из односторонних пределов функции бесконечен.

Горизонтальная асимптота -- это частный случай наклонной.

\opred

Пусть $f$ задана на неограниченном промежутке $X$. Прямая $y=kx+b$ называется наклонной асимптотой кривой $y=f(x)$, если
\[
\lim_{x\to + \infty}(f(x)-kx-b)=0
\]
или
\[
\lim_{x\to - \infty}(f(x)-kx-b)=0
\]

Иногда говорят об асимптоте на бесконечности, не указывая знак. Это означает, что асимптоты на $+\infty$ и $-\infty$ совпадают.

Чтобы выяснить, имеет ли кривая асимптоты и найти $k$ и $b$, разделим равенство

$$
f(x)-kx-b=o(x)
$$

(на $\pm \infty$) на $x$. Получим

$$
k=\frac{f(x)}{x}-\frac{b}{x}-o(x)=\frac{f(x)}{x}-o(x)
$$

\[
k=\lim_{x \to \pm \infty}\frac{f(x)}{x}
\]

\[
b=\lim_{x \to \pm \infty}(f(x)-kx)
\]

Очевидно, что рассуждения верны и в обратную сторону, т. е. прямая $y=kx+b$ будет асимптотой рассматриваемой кривой.

\subsubsection{Замечание}

При $\rho(x) \to \infty$, т. е. при удалении по бесконечной ветви кривой, расстояние $d(M)$ от точки $M$ кривой с координатами $(x; f(x))$ до асимптоты стремится к нулю.

Действительно, пусть $x=x_0$ -- вертикальная асимптота. Тогда $d(M)=|x-x_0|$.
Пусть теперь $y=kx+b$ - наклонная асимптота. Опустим из точки $M$ перпендикуляр $MH$ на асимптоту и перпендикуляр $MB$ на ось $Ox$ и обозначим через $A$ точку пересечения $MB$ с асимптотой. Тогда треугольник $AMH$ - прямоугольный, и катет $MH=d(M)$ в нём меньше гипотенузы $MA$, стремящейся к нулю.

Отметим, что кривая может пересекать свою асимптоту.

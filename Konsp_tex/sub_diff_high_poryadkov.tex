\begin{opred}
Билинейная форма. Функция $A:\R^n\times\R^n\to\R^1,$ если
\\
$$1.A(\alpha x_1+\beta x_2,y)=\alpha A(x_1,y)+\beta A(x_2,y)$$
$$
\forall(\alpha,\beta\in\R^1; x_1,x_2,y\in\R^n)
$$

$$
2.A(x,\aleph y_1 +\beta y_2) = \alpha A(x,y_1)+\beta A(x,y_2)
$$
$$
\forall(\alpha,\beta\in\R^1; x,y_1,y_2\in\R^n)
$$
Если A(x,y) - билинейная форма, определённая на декартовом произведении $\R^n\times\R^n$ по базису $\{e_i\},$ то
$$
A(x,y) = \sum_{i=1}^{n}\sum_{j=1}^{n}x^i y^j ; x=\sum_{i=1}^{n}x^i e_i; y=\sum_{j=1}^{n} y^j e_j
$$
Пусть $a_{ij}=A(e_i,e_j).$ Тогда
$$
A(x,y) = \sum_{i=1}^{n}\sum_{j=1}^{n} a_{ij}x^i y^j
$$
\end{opred}

\begin{opred}
Пусть x=y. Тогда\\
$A(x,x) = \sum_{i=1}^{n}\sum_{j=1}^{n} a_{ij} x^i x^j$ - квадратичная форма, соответствующая билинейной форме $A(x,y)$
\end{opred}

\begin{opred}
Симметрическая билинейная форма $A(x,y)$ и $A(x,x)$ - такие, что
$$
\forall(i,j\in\{1;n\})[a_{ij}=a_{ji}]
$$
\end{opred}

\subsubsection{Пример:}
$<x,y> = \sum_{i=1}^{n}x^i y^i$ - симметрическая билинейная форма.
\\
$||x||^2 = \sum_{i=1}^{n}(x^i)^2$ - симметрическая квадратная форма.
\\
Дифференциал высшего порядка
\\
Пусть $E\subset\R^n$ - открытоею
\\
$f:E\to\R^n$ - дифференцируема на Е, т.е.$\exists(df(x,h))$
\\
Зафиксировав h, получаем скалярную функцию с 1 независимой переменной - х, которая определена на Е и $x+h\in E$
\\
Предположим, что, при фиксированном h, $\varphi(x)=df(x,h)$, как скалярная функция от х, дифференцируема в некоторой точке $x_0\in E\Rightarrow \exists$(линейный функционал $\lambda_{x_0}(x):\R^n\to\R^1)$
\\
$\forall(k\in\R^n:k+x_0\in E)[df(x_0+k,h)-df(x_0,h)=\lambda_{x_0}(h)k+\Omega]$, где $\lambda_{x_0}(h)k$ - линейно по к,
\\
$\Omega = o(||k||), $ т.е. $\frac{\Omega}{||k||}\to 0$ при $||k||\to 0$

\subsubsection{Утверждение:}
Покажем, что $\lambda_{x_0}(h)k$ линейно по h:
\\
т.е. $\forall(h_1,h_2\in\R^n, \alpha\in\R^1)$ выполняется следующее:
$$
1. \lambda_{x_0}(h_1+h_2)k=\lambda_{x_0}(h_1)k+\lambda_{x_0}(h_2)k
$$

$$
2. \lambda_{x_0}(\alpha h_1)k = \alpha\lambda_{x_0}(h_1)k
$$
\dokvo
$(h_1):\lambda_{x_0}(h_1)k+\Omega_1 = df(x_0+k,h_1) - df(x_0,h_1);$
\\
$(h_2):\lambda_{x_0}(h_2)k+\Omega_2 = df(x_0+k,h_2) - df(x_0,h_2);$
\\
$(h_1+h_2):\lambda_{x_0}(h_1+h_2)k+\Omega_3 = df(x_0+k, h_1+h_2) - df(x_0,h_1+h_2),$
где $\Omega_1=o(||k||), \Omega_2 = o(||k||), \Omega_3 = o(||k||)$
\\
$(h_1)+(h_2)-(h_1+h_2):$
\\
$$\lambda_{x_0}(h_1)k+\lambda_{x_0}(h_2)k-\lambda_{x_0}(h_1+h_2)k+(\Omega_1+\Omega_2-\Omega_3) = $$

$$
= df(x_0+k,h_1)-df(x_0,h_1)+df(x_0+k,h_2)-df(x_0,h_2)-df(x_0+k,h_1+h_2)+df(x_0,h_1+h_2)
$$

$$
0 = \lambda_{x_0}(h_1)k+\lambda_{x_0}(h_2)k-\lambda_{x_0}(h_1+h_2)k+(\Omega_1+\Omega_2-\Omega_3) = 
$$
Возьмем $\forall(k_0\in\R^n:k\ne 0).$ Если $k_0=0$, то равенство будет доказано.
\\
В качестве $k=\varepsilon\cdot k_0.$ Тогда, в силу линейности $\lambda_{x_0}:\lambda_{x_0}(h_1)\varepsilon k = \varepsilon\lambda_{x_0}(h_1)k$ Тогда:
$$
= \varepsilon(\lambda_{x_0}(h_1)k_0+\lambda_{x_0}(h_2)k_0-\lambda_{x_0}(h_1+h_2)k_0+\frac{\Omega_1+\Omega_2-\Omega_3}{\varepsilon})
$$
При $\varepsilon\to 0  k=\varepsilon\cdot k_0\to 0\Rightarrow$
\\
$\Rightarrow \frac{\Omega_1+\Omega_2-\Omega_3}{\varepsilon}\to 0$ при $\varepsilon\to 0$
\\
Тогда получаем:
$$
\forall(k_0\ne 0\in\R^n)[\lambda_{x_0}(h_1)k_0+\lambda_{x_0}(h_2)k_0-\lambda(h_1+h_2)k_0]
$$
т.к. $k_0$ может быть любым, то утверждение доказано.
(однородность докажется аналогично)
\dokno
\begin{opred}
Второй дифференциал (дифференциал второго порядка) - квадратичная форма 
$$
\lambda_{x_0}(h)h=d(df(x_0,h)(x_0),h),
$$
соответствующая билинейной форме $\lambda_{x_0}(h)k$ (по аналогии с $f(x_0+h)-f(x_0) = df(x_0,h)+\omega$)
$$
d^2f(x_0,h)=\lambda_{x_0}(h)(h).
$$
\end{opred}

\begin{opred}
Вторая производная функции f в $x_0$ - билинейное отображение, значением которого в точке (h,h) является второй дифференциал 
$$
f''(x_0)
$$
\end{opred}
Таким образом, $d^2f(x_0,h)=f''(x_0)(h,h).$
\\
По аналогии со скалярным случаем:
$$
f''(x_0)(h,h) = f''(x_0)h^2
$$
Так как 2 дифференциал - это квадратичная форма, то он представляется в виде:
$$
d^2f(x_0,h)=|\sum_{i=1}^{n}\sum_{j=1}^{n} a_{ij} h^i h^j|
$$
здесь $a_{ij}$ зависит от $x_0$
\\
Найдем $a_{ij}(x_0):$
$df(x,h)=\sum_{i=1}^{n}\frac{\delta f}{\delta x^2}(h^i)$
\\
В качестве h возьмем $h^0=(0,...,0,1,0...)$
\\
Подставим:
$$
df(x,h^0)=\frac{\delta f}{\delta x^k}(x)
$$
$df(x,h^0)$ - диф-ем как функцию от х в $x_0$
\\
Тогда $df(x,h^0)$ имеет в $x_0$ все частные производные. Значит,
$$
\exists(\frac{\delta}{\delta x^j}(\frac{\delta f}{\delta x^k})(x_0) = \frac{\delta^2 f}{\delta x^j \delta x^k}(x_0))
$$

$d^2 f(x,h^0) = \sum_{i=1}^{n}\frac{\delta}{\delta x^i}(\sum_{j=1}^{n}\frac{\delta f}{\delta x^j}(x)\cdot h^j )(x_0)h^i=$
\\
$\sum_{i=1}^{n}\sum_{j=1}^{n}\frac{\delta^2 f}{\delta x^i \delta x^j}(x_0)h^i h^j\Rightarrow$
\\
$\Rightarrow a_{ij}=\frac{\delta^2 f}{\delta x^i \delta x^j}(x_0)$
\\
Таким образом, если функция дважды дифференцируема в $x_0$, тогда второй дифференциал представляется в виде:
$$
d^2 f(x_0,h) = \sum_{i=1}^{n}\sum_{j=1}^{n}\frac{\delta^2 f}{\delta x^i \delta x^j}(x_0)h^i h^j
$$
Если, по аналогии со скалярным случаем, приращение обозначим как dx, то:
$$
d^2 f(x_0,dx) = \\sum_{i=1}^{n}\sum_{j=1}^{n}\frac{\delta^2 f}{\delta x^i \delta x^j}(x_0)dx^i dx^j
$$

Позже будет доказано, что, если функция дважды диф-ема в $x_0$, то верно, что
$$
\frac{\delta^2 f}{\delta x^i \delta x^j}(x_0)=\frac{\delta^2 f}{\delta x^j \delta x^i}(x_0)
$$
Если предположить, что любая частная производная f в $x_0$ - непрерывна, то функция f дважды диф-ема и имеет место последняя формула.
\subsubsection{Пример:}
$f:\R^2\to\R^1  f(x^1,x^2) = (x^1)^2\cdot(x^2)$
\\
$\frac{\delta f}{\delta x^1} = 2x^1 x^2; \frac{\delta f}{\delta x^2}=(x^1)^2; \frac{\delta^2 f}{d(x^1)^2} = 2x^2;\frac{\delta^2 f}{\delta (x^2)^2}=0;\frac{\delta^2 f}{\delta x^1 \delta x^2}=2x^1$

\begin{opred}
Трилинейная форма - отображение декартового произведения 3-х множтелей, которые являются линейными по каждому аргументу при фиксированных двух остальных:
$$
\forall(\alpha,\beta\in\R^1)\forall(x_1,x_2,y,z\in\R^n)
$$

$$
[A(\alpha x_1+\beta x_2,y,z) = \alpha A(x_1,y,z)+\beta A(x_2,y,z)]
$$
\end{opred}
Если $\forall(x\in E)\exists(d^2 f(x,h)),$ то дифференциал в т. $x_0\in E - $есть трилинейная форма от k,h,m:
\\
	k - приращение по x в $df(x,k)$
	\\
	h - приращение по x в $d^2 f(x,h)$
	\\
	m - приращение по y в $df(y,m)$
\\
\begin{opred}
Третий дифференциал или дифференциал третьего порядка функции f в $x_0$ - однородная форма, которая получается с трилинейной формы при k=h=m.
\\
$d^3 f(x_0,h)$
\end{opred}

\begin{opred}
Производная третьего порядка функции f в $x_0$ - трилинейное отображение, значением которого в т. (h,h,h) будет третий дифференциал.
\\
$f'''(x_0)$
\end{opred}
Таким образом, $d^3 f(x_0,h)=f'''(x_0)(h,h,h)$
\\
По аналогии со скалярной функцией:
$$
d^3 f(x_0,h) = f'''(x_0)(h,h,h)= f'''(x_0)h^3
$$
Продолжая, по индукции можно определить понятия дифференциала и производной любого порядка для скалярной функции векторного аргумента. Также, используя метод индукции, можно доказать, что:
$$
d^k f(x_0,h)=\sum_{i_1,...,i_{k+1}}^{n}\frac{\delta^k f}{\delta x^{i1}....\delta x^{ik}}(x_0)\cdot h^{i1}...h^{ik}
$$
или, если обозначить h за dx, то:
$$
d^k f(x_0,h)=\sum_{i_1,...,i_{k+1}}^{n}\frac{\delta^k f}{\delta x^{i1}....\delta x^{ik}}(x_0)\cdot dx^{i1}...dx^{ik}
$$

\begin{teorema}
О дифференцируемости сложной функции.
\\
Пусть $E\subset\R^n,$ а $f:E\to\R_1$ - p штук раз диф-ема на Е. Пусть $\varphi^i:H\to\R^n$, где $H\subset\R^n$ - открытое.
\\
Притом $\forall(t\in H)[\varphi^1(t),...,\varphi^n(t)\in H]$ и $\forall(i\in\{1;n\})$ $\varphi^i$ - р штук раз диф-ема в т $t_0\in H$ (или на Н)
\\
Частный случай теоремы: $f:\R^n\to\R^k \forall(k\in\N)$
\\
Например: $f:\R^1\to\R^2  f(x)=(x,x^2+4)$
\end{teorema}
\dokvo
Второй дифференциал и все последущие - не инвариантны.
\\
Пусть (без ограничения общности) $f:\R^n\to\R^2$
\\
$f(x^1,x^2)$ - дважды диф-ема на $E\subset\R^2$
\\
$x^1=\varphi^1(t^1,...,t^m); x^2=\varphi^2(t^1,...,t^m)$
\\
$\varphi^1$ и $\varphi^2$ - дважды диф-емы на $H\subset\R^m.$
\\
В силу инвариантности формы первого дифференциала, df(x,dx) имеют одинаковую формулу независимо от того зависима х или независима.
$$
df(x,dx)=\frac{\delta f}{\delta x^1}(x)dx^1+\frac{\delta f}{\delta x^2}(x)dx^2,
$$

$$
dx^1 = \sum_{i=1}^{m}\frac{\delta\varphi^1}{\delta t^i}dt^i, dx^2 = \sum_{i=1}^{m}\frac{\delta\varphi^2}{\delta t^i}dt^i
$$

$$
d^2 f(x,dx) = d((\frac{\delta f}{\delta x^1}(x)dx^1+\frac{\delta f}{\delta x^2}(x)dx^2),dx)=
$$

$$
=\frac{\delta^2 f}{\delta x^2 \delta x^1}(dx^1)^2+\frac{\delta^2 f}{\delta x^1 \delta x^2}(dx^1 dx^2)+\frac{\delta^2 f}{\delta x^2 \delta x^1}(dx^2 dx^1)+\frac{\delta^2 f}{\delta x^2 \delta x^2}(dx^2)^2+\frac{\delta f}{\delta x^1}(x)d^2x^1+\frac{\delta f}{\delta x^2}(x)d^2x^2=
$$

$$
\frac{\delta^2 f}{\delta (x^1)^2}(dx^1)2+2\frac{\delta^2 f}{\delta x^1 \delta x^2}(dx^1 dx^2)+\frac{\delta^2 f}{\delta(x^2)^2}(dx^2)^2+[\frac{\delta f}{\delta x^1}(x)d^2 x^1 + \frac{\delta f}{\delta x^2}(x)d^2 x^2]^*
$$
Если $x^1,x^2$ - независимые переменные, то $[...]^* = 0$ - получили обычную формулу для $d^2 f$
\\
Значит, второй дифференциал не инвариантен.
\dokno




\subsubsection{Теорема.}
Пусть $F$ -- первообразная для $f$ -- непрерывной функции на промежутке $T$, т. е.
$$\int f(t)dt=F(t)+C$$
и на промежутке $X$ задано $\varphi:X\to T$ -- непрерывное дифференцируемое отображение.

Тогда на промежутке $X$
$$\int f(\varphi(x))\cdot \varphi '(x) dx=F(\varphi(x))+C$$
Т. е.
$$\int f(\varphi(x))\cdot d\varphi(x)=F(\varphi(x))+C$$

\dokvo
$$(F(\varphi(x))+C)'=f(\varphi(x))\cdot \varphi'(x)$$
\dokno

\subsubsection{Пример.}

$$\int x e^{x^2} dx = 
\left<\begin{array}{c}
t=x^2 \\
dt=2xdx
\end{array}\right>= %$$ $$=
\frac{1}{2}\int e^t dt = \frac{1}{2} e^t +C = \frac{1}{2}e^{x^2}+C
$$

\subsubsection{Пример.}

$$\int \cos^2 x \sin x dx = 
\left<\begin{array}{c}
t=\sin x \\
dt=-\cos x
\end{array}\right>=$$ $$=
-\int t^2 dt = -\frac{t^3}{3}+C = -\frac{\cos^3 x}{3} +C
$$

\subsubsection{Следствие.}
Если $F'(x)=f(x)$ и $\{a;b\}\in\R$, то
$$\int f(ax+b)dx=\frac{1}{a}F(ax+b)+C$$

\subsubsection{Пример.}

$$\int\cos(7x+3)dx=-\frac{1}{7}\sin(7x+3)+C$$

\subsubsection{Замечание 1.}
Полезно помнить следующие интегралы:

$$\int \frac{g'(x)}{g(x)} dx = 
\left<\begin{array}{c}
t=g(x) \\
dt=g'(x)dx
\end{array}\right>=$$ $$=
\int \frac{dt}{t}=\ln|g(x)|+C
$$

$$\int \frac{g'(x)}{\sqrt{g(x)}} dx = 
\left<\begin{array}{c}
t=g(x) \\
dt=g'(x)dx
\end{array}\right>=$$ $$=
\int \frac{dt}{\sqrt{t}}=2\sqrt{g(x)}+C
$$

\subsubsection{Замечание 2.}
Замену переменной под знаком неопределённого интеграла часто производят иначе: вместо того, чтобы принимать за новую переменную $t$ некоторую функцию $f(x)$, рассматривают $x$ как дифференцируемую функцию от $z$, т. е. $x=\psi(z)$. Тогда
$$\int f(x)dx=\int f(\psi(x))\psi'(z)dz$$
Однако при применении этого метода нужно убедиться, что существует обратная функция $\psi^{-1}(x)=z$, позволяющая вернуться от $z$ к исходной переменной $x$.

\subsubsection{Пример.}
$$\int \sqrt{1-x^2}dx =
\left<\begin{array}{c}
t=\sin z \\
|x|\leq 1; |z|\leq \frac{\pi}{2}\\
dx=\cos z dz
\end{array}\right>=$$ $$=
\int\sqrt{1-\sin^2 z} \cos z dz=$$$$=
\int\frac{1+\cos 2z}{2}=\frac{1}{2}\int dz +\frac{1}{2}\cdot \frac{1}{2} \sin 2z +C=$$$$=
\frac{\arcsin x}{2}+\frac{\sin(2\arcsin x)}{4}+C=\frac{\arcsin x + x\sqrt{1-x^2}}{2}+C
$$

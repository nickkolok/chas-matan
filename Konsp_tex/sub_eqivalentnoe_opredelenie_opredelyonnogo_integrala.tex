И снова начнём со вспомогательного определения:

\opred
Последовательность разбиений $\{T_n\}$ отрезка $[a;b]$ называется неограниченно измельчающейся, если 
$$\lim_{n\to\infty}d(T_n) = 0$$

Проницательный читатель наверняка предположил, что раз существует определение определённого интеграла (\ref{def_opred_integral_1}), аналогичное определению предела функции по Коши, то существует и определение, аналогичное определению предела по Гейне. Сформулируем его:

\begin{opr}\label{eqiv_opr_opr_intl}
Пусть $f:[a;b]\to \R$. Функция $f$ называется интегрируемой по Риману, если
\begin{equation}\label{def_opred_integral_2}
\exists(J\in\R)\forall\left(\left\{\left(T_n,\xi^{(n)}\right)\right\}\right)
\left[\lim_{n\to\infty}d(T_n) = 0 \Rightarrow \lim_{n\to\infty}S\left(f,\left(T_n,\xi^{(n)}\right)\right)=J\right]
\end{equation}
\end{opr}

\subsubsection{Теорема.}
Определения (\ref{def_opred_integral_1}) и (\ref{def_opred_integral_2}) эквивалентны.

Докажем сначала, что (\ref{def_opred_integral_1}) $\Rightarrow$ (\ref{def_opred_integral_2})

\dokvo
Пусть $$J=\intl_a^b f(x) dx $$ в смысле определения (\ref{def_opred_integral_1}).

Зафиксируем любую бесконечно измельчающуюся последовательность разбиений $\left\{\left( T_n, \xi^{(n)}\right)\right\}$. Тогда $d(T_n)\to 0$, и, следовательно,
\begin{equation}\label{dokvo_eqiuv_def_opred_int_1}
\forall(\delta>0)\exists(n_0\in\N)\forall(n\geq n_0)[d(T_n)<\delta]
\end{equation}

С другой стороны, по определению (\ref{def_opred_integral_1}),
$$
\forall(\varepsilon>0)\exists(\delta>0)\forall((T,\xi))[d(T)<\delta \Rightarrow |S(f,(T,\xi))-J|<\varepsilon]
$$

Зафиксировав $\varepsilon$ и найдя из этого условия $\delta$, с учётом (\ref{dokvo_eqiuv_def_opred_int_1}) получим:
$$
\forall(\varepsilon>0)\exists(n_0\in\N)\forall(n\geq n_0)[d(T_n)<\delta]
$$
Следовательно,
$$
\forall(\varepsilon>0)\exists(n_0\in\N)\forall(n\geq n_0)\left[\left|S\left(f,\left(T_n,\xi^{(n)}\right)\right)-J\right|<\varepsilon\right]
$$
Из этого условия непосредственно следует, что $J=\intl_a^b f(x) dx$ в смысле определения (\ref{def_opred_integral_2})

Докажем теперь, что из выполнения определения (\ref{def_opred_integral_2}) следует выполнение (\ref{def_opred_integral_1})

\dokvo
\pp: пусть определение (\ref{def_opred_integral_2}) выполнено, а определение (\ref{def_opred_integral_1}) - нет, т. е. 
$$
\exists(\varepsilon>0)\forall(\delta>0)\exists((T,\xi))[d(T)<\delta \cap |S(f,(T,\xi))-J|\geq \varepsilon]
$$
Зафиксируем найденное $\varepsilon$ и будем брать $\delta$ из последовательности $\left\{\frac{1}{n}\right\}$. Тогда разбиения $\left(T_n,\xi^{(n)}\right)$ образуют бесконечно измельчающуюся последовательность. Но эта последовательность не сходится к $J$, т. к.
$$\left|S\left(f,\left(T_n,\xi^{(n)}\right)\right)-J\right|\geq \varepsilon$$
Таким образом, определение (\ref{def_opred_integral_2}) не выполнено. Получили противоречие, следовательно, наше допущение о том, что определение (\ref{def_opred_integral_1}) не выполнено -- неверно.
Эквивалентность определений доказана.

\dokno

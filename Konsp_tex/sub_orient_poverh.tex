\opred

Точка $M_0$ поверхности $S$ внутренняя, если существует окрестность точки $M_0$ такая, что множество точек окрестности, не являющихся точками поверхности представляет собой несвязное множество.

\opred

Точки поверхности не являющиеся внутренними - граничные точки поверхности.
\\
Рассмотрим поверхность $S$; $M_0$ - внутренняя точка. Будем считать, что в некоторой внутренней точке $M_0$ гладкой поверхности $S$ выбрано одно из двух направлений нормали.

\opred

Если при обходе вдоль дугового замкнутого контура $\Gamma$, лежащего на $S$ и, не имеющего общих точек с границей этой поверхности, нормаль, непрерывно изменяясь вдоль $\Gamma$, по возвращению в точку $M_0 \in \Gamma$ вернется к первоначальному направлению, то поверхность $S$ - ориентированная (двусторонняя).

Выбор одного из направлений нормали в какой-либо внутренней точке ориентированной поверхности определяет сторону поверхности.
\\
Примеры:
\\
Двусторонние поверхности:
\\
Полусфера, плоскость, эллипсоид, однополосный гиперболоид.
\\
Односторонняя поверхность - лист Мёбиуса.

Пусть $S$ - двусторонняя поверхность (есть направление нормали). Пусть есть две точки $M_0$ и $M_1 \in S$ и гладкий контур $\Gamma_1$, соединяющий их и не пересекающий границы $S$. Тогда при перемещении из одной точки в другую вдоль $\Gamma$, во вторую точку мы придем с тем же самым направлением нормали.

Если, приходя в $M_1$, по двум разным путям $\Gamma_1$ и $\Gamma_2$ мы получили бы разное направление нормали, то это бы привело к тому, что, двигаясь по замкнутому контуру $M_1 \Gamma_1 M_0 \Gamma_2 M_1$, мы бы пришли в $M_1$ с другим направлением нормали, что противоречит условию.

Таким образом, выбор направления нормали в одной точке однозначно определяет выбор направления нормали на всей $S$. 

\opred

Совокупность точек поверхности с приписанными направлениями нормали называется определённой стороной поверхности.

Пусть $S$ - незамкнутая, гладкая двусторонняя поверхность, ограниченная простым контуром. Обозначим определённое направление обхода в качестве "$+$" , если наблюдатель, двигаясь по контуру видит внутреннюю часть поверхности слева. Противоположное направление - "$-$".

Обозначим за $S^+$ сторону поверхности с "$+$" направлением обхода контура, за $S^-$ - сторону с "$-$" направлением обхода контура.

Рассмотрим негладкую поверхность $S$(кубик). 

Разобьем гладкую поверхность $S$ на две части: $S^+$ И $S^-$: сфера - тут все ясно, тор - проблема.

Но мы ограничимся случаями, когда поверхность разделяется с помощью некоторого гладкого контура и указываем в качестве "$+$" ориентации - внешнюю сторону поверхности, а в качестве "$-$" - противоположную.

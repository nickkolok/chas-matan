\subsubsection{Метод.}
Пусть $u(x)$ и $v(x)$ на некотором промежутке $X$ -- дифференцируемые функции. Тогда
$$\int u(x)\cdot v'(x)dx=u(x)\cdot v(x) - \int v(x) \cdot u'(x)dx$$

Т. е., перейдя к дифференциалам функций,
$$\int udv=uv- \int vdu$$

\dokvo
Нам известна формула дифференцирования произведения:
$$(u(x)\cdot v(x))'=u'(x)v(x)+v'(x)u(x)$$
Интегрируем её:
$$u(x)\cdot v(x)'=\int u'(x)v(x) dx +\int v'(x)u(x) dx$$
И переносим один из интегралов в левую часть:
$$u(x)\cdot v(x) - \int v(x) \cdot u'(x)dx=\int u(x)\cdot v'(x)dx$$

\dokno

\subsubsection{Замечание 1.}

При использовании формулы интегрирования по частям подынтегральную функцию нужно представить в виде произведения одной функции на дифференциал другой.
Делают так, чтобы интеграл $\int vdu$ оказался проще, чем интеграл $\int udv$.
Иногда формулу интегрирования по частям приходится применять несколько раз.

\subsubsection{Замечание 2.}

Функция $v$ по $dv$ восстанавливается, вообще говоря, неоднозначно, с точностью до постоянного слагаемого. Его можно считать равным нулю.

\dokvo
Пусть по дифференциалу $dv$ нашлись функции $v_0$ и $v_0+C$. На левую часть, т. е. $\int udv$, $C$ не влияет, т. к. $d(v_0)=d(v_0+C)$. Рассмотрим правую часть:
$$
u\cdot (v_0+C) - \int (v_0+C)du=
uv_0+uC-\int v_0 du - C\int du=$$$$=
uv_0+uC-\int v_0 du - Cu=
uv_0-\int v_0 du
$$
\dokno

\subsubsection{Замечание 3.}
Интегрирование по частям особенно эффективно при интегрировании, если:

а) $u(x)=P_n(x)$, т. е. многочлен от $x$, а $v'(x) \in \{e^x,\sin x,\cos x\}$

б) $u(x) \in \{\ln x, \arctg x \}$, $v'(x)=P_n(x)$

\subsubsection{Пример.}

$$\int x^2 e^x dx = \int \left(\frac{x^3}{3}\right)'e^x dx=$$ $$=
\left<\begin{array}{c|c}
u=x^2 & du=2xdx \\
dv=e^x dx & v=e^x
\end{array}\right>=$$ $$=
x^2 e^x - 2\int e^x \cdot x dx=$$ $$=
\left<\begin{array}{c|c}
u=x & du=dx \\
dv=e^x dx & v=e^x
\end{array}\right>=$$ $$=
x^2 e^x-2\left(e^x \cdot x - \int e^x dx \right)=x^2 e^x - 2 x e^x + 2 e^x +C
$$



\subsubsection{Теорема.}

Для того, чтобы число $a \in \R$ было верхним пределом последовательности $\{x_n\}$, необходимо и достаточно выполнения следующих двух условий:

1)$\forall(\epsilon >0 )\exists(n_0 \in \N)\forall(n \geq n_0)[x_n < a + \epsilon]$

2)$\forall(\epsilon > 0)\forall(m   \in \N)\exists(n \geq m  )[x_n > a - \epsilon]$

\subsubsection{Замечание.}
Условие (1) означает, что количество членов последовательности, б\`{о}льших $a+\epsilon$, конечно.

Условие (2) означает, что количество членов подпоследовательности, б\`{о}льших $a-\epsilon$, бесконечно.

\paragraph{Доказательство.}

\paragraph{Необходимость.}
Дано:
$a \in \R$~--- верхний предел последовательности $\{x_n\}$, т.е. наибольший из её частичных пределов.

Доказать: выполнение условий (1) и (2).

(1): Предположим противное, т.е.
$$
	\exists(\epsilon_0 >0 )\forall(k \in \N)\exists(n_k \geq k)[x_{n_k} \geq a + \epsilon]
	.
$$
Тогда последовательность $\{x_{n_k}\}$ целиком лежит в луче $[a + \epsilon; +\infty)$,
а в случае ограниченности последовательности $\{x_n\}$~--- и вовсе в некотором отрезке $[a + \epsilon; M]$.
Значит, последовательность $\{x_{n_k}\}$ имеет частичный предел, принадлежащий $[a + \epsilon; +\infty]$.
Но частичный предел подпоследовательности $\{x_{n_k}\}$ является частичным пределом последовательности $\{x_n\}$,
следовательно, $a$ не является верхним пределом.

(2): непосредственно следует из того, что $a$~--- частичный предел.

\paragraph{Достаточность.}
Дано: выполнение условий (1) и (2).

Доказать: $a \in \R$~--- верхний предел последовательности $\{x_n\}$, т.е. наибольший из её частичных пределов.

Доказываем в два этапа: сначала докажем, что $a$~--- частичный предел,
потом~--- что он наибольший.

Этап 1.
Положим $\varepsilon = 1/k$, $k\in\mathbb{N}$.
Тогда из условия 1) при подстановке такого $\varepsilon$ следует, что
$$
	\exists(m_k \in \N)\forall(n \geq m_k)[x_n < a + 1/k]
	.
$$
В условии 2) положим $\varepsilon = 1/k$, $m=\max\{m_k,n_{k-1}\}$ (полагая $n_0=0$):
$$
	\exists(n_k > \max\{m_k,n_{k-1}\}  )[x_{n_k} > a - 1/k]
	.
$$
Тогда подпоследовательность $x_{n_k}$ сходится к $a$,
следовательно, $a$~--- частичный предел последовательности $x_n$.

Этап 2.
Покажем, что $a$~--- наибольший из частичных пределов.
Предположим противное, т.е. существует частичный предел $b$, при этом $b>a$.
Положим $\varepsilon = (b-a)/3$.
С одной стороны, в силу условия 1) в луче $[a+\varepsilon; +\infty)$ лежит
лишь конечное число членов последовательности $\{x_n\}$.
С другой стороны, в силу определения частичного предела
в интервале $(b-\varepsilon; b+\varepsilon)$
лежит бесконечноечисло членов последовательности $\{x_n\}$.
Но в силу выбора $\varepsilon$ имеем
$(b-\varepsilon; b+\varepsilon) \subset \lbrack a+\varepsilon; +\infty)$.
Полученное противоречие завершает доказательство.


\par

Аналогично формулируется характеристическое свойство нижнего предела:

\subsubsection{Теорема.}

Для того, чтобы число $a \in \R$ было нижним пределом последовательности $\{x_n\}$, необходимо и достаточно выполнения следующих двух условий:

1)$\forall(\epsilon >0 )\exists(n_0 \in \N)\forall(n \geq n_0)[x_n > a - \epsilon]$

2)$\forall(\epsilon > 0)\forall(m   \in \N)\exists(n \geq m  )[x_n < a + \epsilon]$

\subsubsection{Замечание.}
Условие (1) означает, что количество членов последовательности, меньших $a-\epsilon$, конечно.

Условие (2) означает, что количество членов подпоследовательности, меньших $a+\epsilon$, бесконечно.



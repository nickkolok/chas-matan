Зачастую бывает, что интеграл \ref{nesobstv_integr_po_neogr_promezh_raven_sum} расходится, так как расходятся оба его ``слагаемых''.
Тем не менее, например, для $f(x)=\sin x$ и других нечётных функций, идеологически было бы верно каким-либо образом связать интеграл \ref{nesobstv_integr_infty_infty} с каким-либо числом.

\begin{opr}
Пусть $f:\R\to\R$, $\forall(\xi\in\R,\eta\in\R)\bigl[f\in R[\xi;\eta]\bigr]$.
Пусть существует предел
\begin{equation}\label{predel_glavn_znach_v_smysle_Koshi}
\lim_{\xi\to+\infty} \intl_{-\xi}^\xi f(x) dx
\end{equation}
Тогда этот предел называют главным значением интеграла по числовой прямой в смысле Коши:
\begin{equation}
\lim_{\xi\to+\infty} \intl_{-\xi}^\xi f(x) dx = v.p.\intl^{+\infty}_{-\infty} f(x) dx
\end{equation}
\end{opr}
\begin{zamech}
Мы потребовали существования предела \ref{predel_glavn_znach_v_smysle_Koshi}, и поэтому нам не нужно делать оговорок о мнемоничности формулы, как в случае с формулами \ref{nesobstv_integr_a_infty_raven_lim} и \ref{nesobstv_integr_infty_a_raven_lim}.
\end{zamech}
\begin{zamech}
Сокращение ``v. p.'' происходит от французских слов ``valeur principale'', что и означает ``главное значение'' (но прилагательное стоит после существительного).
\end{zamech}
\begin{zamech}
Из определения \ref{opr_nesobstv_integr_infty_infty} при $a=0$ вытекает, что если интеграл
$$
\intl^{+\infty}_{-\infty} f(x) dx
$$
сходится, то 
$$
v.p.\intl^{+\infty}_{-\infty} f(x) dx = \intl^{+\infty}_{-\infty} f(x) dx
$$
Обратно неверно (контрпример уже упоминался выше: $f(x)=\sin x$).
\end{zamech}
%TODO:пример с синусом


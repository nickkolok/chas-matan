Здесь и далее мы будем иногда без особого предупреждения использовать различные нормы в $\R^n$.
Внимательному читателю не составит труда вспомнить, что все они эквивалентны в силу конечномерности $\R^n$ (а невнимательному мы только что напомнили).

Дадим классическое определение непрерывности по Коши:
\begin{opr}
Пусть $f:A\subset\R^n\to\R^m$.
$f$ непрерывно в точке $x_0\in A$, если
$$
\forall(\varepsilon>0)\exists(\delta>0)\forall(x:0<|x-x_0|<\delta)[|f(x_0)-f(x)|<\varepsilon]
$$
\end{opr}

\begin{utverzhd}
Определение через непрерывность координатных функций (определение \ref{opred_predela_Rn_Rm_nepr}) и приведённое выше определения эквивалентны.
\end{utverzhd}

\dokvo
Положим для $y=(y^1, ... , y^m) |y|=\max\limits_{i}|y^i|$. 
Тогда
\begin{multline*}
\forall(\varepsilon>0)\exists(\delta>0)\forall(x:0<|x-x_0|<\delta)[|f(x_0)-f(x)|<\varepsilon]
\\\Leftrightarrow
\forall(\varepsilon>0)\exists(\delta>0)\forall(x:0<|x-x_0|<\delta)[\max\limits_{i}|f^i(x_0)-f^i(x)|<\varepsilon]
\Leftrightarrow\\
\forall(i\in\{1,...,m\})\forall(\varepsilon>0)\exists(\delta>0)\forall(x:0<|x-x_0|<\delta)[|f^i(x_0)-f^i(x)|<\varepsilon]
\end{multline*}
Предыдущая строка и означает непрерывность координатных функций.
\dokno

Теперь, следуя порядку изложения предыдущих разделов, дадим определения через окрестности:

\begin{opr}
Пусть $f:A\subset\R^n\to\R^m$.
$f$ непрерывно в точке $x_0\in A$, если
$$
\forall(\varepsilon>0)\exists(\delta>0)[x\in\mathring{U}_\delta(x_0)\cap A \Rightarrow f(x)\in U_\varepsilon(f(x_0))]
$$
\end{opr}

\begin{opr}
Пусть $f:A\subset\R^n\to\R^m$.
$f$ непрерывно в точке $x_0\in A$, если
$$
\forall(V(f(x_0))\exists(\mathring{U}(x_0))[f(\mathring{U}(x_0)\cap A)\in V(f(x_0))]
$$
\end{opr}

Наконец, перепишем то же самое в предельной форме:

\begin{opr}
Пусть $f:A\subset\R^n\to\R^m$.
$f$ непрерывно в точке $x_0\in A$, если
$$
\lim\limits_{x\to x_0, x\in A} f(x) = f(x_0)
$$
\end{opr}

Теперь сформулируем определение непрерывности по Гейне:

\begin{opr}
Пусть $f:A\subset\R^n\to\R^m$.
$f$ непрерывно в точке $x_0\in A$, если
$$
\forall(\{x_k\}\subset A\\\{x_0\}:\{x_k\}\to x_0)[\{f(x_k)\}\to f(x_0)]
$$
\end{opr}

Его эквивалентность определению \ref{opred_predela_Rn_Rm_nepr} доказывается аналогично --- через переход к координатным функциям.


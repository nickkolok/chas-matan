\begin{opr}
	Число $b$ называется пределом функции $f(x)$ при $x$, стремящемся к $a$, если
	\begin{equation}
		\forall(\varepsilon > 0)\exists(\delta>0)\forall(x: 0<|x-a|<\delta)[|f(x)-b|<\varepsilon]
		.
	\end{equation}
	Пишут:
	\begin{equation}
		\lim_{x\to a} f(x) = b
		.
	\end{equation}
\end{opr}
Сама функция $f$ может и не быть определена в точке $a$, например
\begin{equation}
	\lim_{x\to 0} \frac{\sin x}{x} = 0
	,
\end{equation}

\begin{equation}
	\lim_{x\to 1} \frac{x^2-1}{x-1} = 2
	.
\end{equation}

Вообще говоря, предела у функции может и не существовать.
Например, не существует предела
\begin{equation}
	\lim_{x\to 0} \frac{|x|}{x}
	.
\end{equation}

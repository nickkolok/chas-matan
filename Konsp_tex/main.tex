\documentclass[a4paper,14pt]{report} %размер бумаги устанавливаем А4, шрифт 12пунктов
\usepackage[T2A]{fontenc}
\usepackage[utf8]{inputenc}
\usepackage[english,russian]{babel} %используем русский и английский языки с переносами
\usepackage{amssymb,amsfonts,amsmath,mathtext,cite,enumerate,float} %подключаем нужные пакеты расширений
\usepackage[pdftex,unicode]{hyperref}
\usepackage{indentfirst} % включить отступ у первого абзаца
\usepackage[dvips]{graphicx} %хотим вставлять рисунки?
\graphicspath{{illustr/}}%путь к рисункам

\makeatletter
\renewcommand{\@biblabel}[1]{#1.} % Заменяем библиографию с квадратных скобок на точку:
\makeatother %Смысл этих трёх строчек мне непонятен, но поверим "Запискам дебианщика"

\usepackage{geometry} % Меняем поля страницы. 
\geometry{left=1cm}% левое поле
\geometry{right=1cm}% правое поле
\geometry{top=1cm}% верхнее поле
\geometry{bottom=2cm}% нижнее поле

\renewcommand{\theenumi}{\arabic{enumi}}% Меняем везде перечисления на цифра.цифра
\renewcommand{\labelenumi}{\arabic{enumi}}% Меняем везде перечисления на цифра.цифра
\renewcommand{\theenumii}{.\arabic{enumii}}% Меняем везде перечисления на цифра.цифра
\renewcommand{\labelenumii}{\arabic{enumi}.\arabic{enumii}.}% Меняем везде перечисления на цифра.цифра
\renewcommand{\theenumiii}{.\arabic{enumiii}}% Меняем везде перечисления на цифра.цифра
\renewcommand{\labelenumiii}{\arabic{enumi}.\arabic{enumii}.\arabic{enumiii}.}% Меняем везде перечисления на цифра.цифра
\LARGE

\begin{document}
\newcommand{\pp}{Предположим противное}
\tableofcontents % это оглавление, которое генерируется автоматически

\LARGE

\chapter{Введение в анализ}
\section{Элементарные сведения из логики и теории множеств}
\subsection{Высказывания, предикаты связки}
\subsection{Кванторы}
\subsection{Множества, равенство двух множеств, подмножества}
\subsection{Простейшие операции над множествами}
\subsection{Принцип двойственности}
\subsection{Понятие счётного множества}
...

\section{Теория вещественных чисел}
\subsection{Множество рациональных чисел и его свойства}
\subsection{Вещественные числа, основные свойства вещественных чисел}
\subsection{Промежутки и их виды}
\subsection{Основные леммы теории вещественных чисел}
...

\section{Ограниченное множество, границы}
\subsection{Границы множества}
\subsection{Существование точной верхней границы у ограниченного сверху множества}
\subsection{Сечения в множестве рациональных чисел}
\subsection{Свойства $\sup$ и $\inf$}
\subsection{Отделимость множеств, лемма о системе вложенных отрезков}
\subsection{Лемма о последовательности стягивающихся отрезков}
...

\section{Отображения, функции}
\subsection{Отображения, виды отображений и т. д.}
\subsection{Вещественные функции}
...

\section{Предел последовательности}
\subsection{Последовательность элементов множества, числовая последовательность, определения предела числовой последовательноcти и бесконечно малой последовательности}
\subsection{Единственность предела последовательности}
\subsection{Подпоследовательности, связь пределов последовательности и подпоследовательности}
\subsection{Лемма о двух милиционерах}
\subsection{Основные теоремы о пределах последовательности}
\subsection{Понятие бесконечно большой последовательности}
\subsection{Монотонные последовательности, критерий существования предела монотонной послед}
\subsection{Существование предела последовательности $(1+1/n)^n$, число $e$ }
...

\section{Понятие предельной точки числового множества, теорема Больцано-Вейерштрасса, критерий Коши}
\subsection{Предельная точка множества}
\subsection{Теорема о последовательности, сходящейся к предельной точке}
\subsection{Теорема Больцано-Вейерштрасса}
\subsubsection{Теорема.}

Любое бесконечное ограниченное множество вещественных чисел имеет хотя бы одну предельную точку.

\mnemo

Название теоремы удобно запоминать по первым буквам прилагательных:

<<\textbf{Б}есконечное \textbf{о}граниченное множество \textbf{в}ещественных чисел>>

<<\textbf{Бо}льцано-\textbf{В}ейерштрасса>>

\subsection{Критерий Коши}
...

\section{Верхний и нижний пределы последовательности}
\subsection{Понятие расширенной числовой прямой, понятие бесконечных пределов}
\subsection{Понятие частичных верхних и нижних пределов последовательности. Теорема о существовании у каждой последовательности ее верхнего и нижнего предела}
\subsection{Характеристические свойства верхнего и нижнего предела последовательности}
\subsubsection{Теорема.}

Для того, чтобы число $a \in \R$ было верхним пределом последовательности $\{x_n\}$, необходимо и достаточно выполнения следующих двух условий:

1)$\forall(\epsilon >0 )\exists(n_0 \in \N)\forall(n \geq n_0)[x_n < a + \epsilon]$

2)$\forall(\epsilon > 0)\forall(m   \in \N)\exists(n \geq m  )[x_n > a - \epsilon]$

\subsubsection{Замечание.}
Условие (1) означает, что количество членов последовательности, б\`{о}льших $a+\epsilon$, конечно.

Условие (2) означает, что количество членов подпоследовательности, б\`{о}льших $a-\epsilon$, бесконечно.

\paragraph{Доказательство.}

\paragraph{Необходимость.}
Дано:
$a \in \R$~--- верхний предел последовательности $\{x_n\}$, т.е. наибольший из её частичных пределов.

Доказать: выполнение условий (1) и (2).

(1): Предположим противное, т.е.
$$
	\exists(\epsilon_0 >0 )\forall(k \in \N)\exists(n_k \geq k)[x_{n_k} \geq a + \epsilon]
	.
$$
Тогда последовательность $\{x_{n_k}\}$ целиком лежит в луче $[a + \epsilon; +\infty)$,
а в случае ограниченности последовательности $\{x_n\}$~--- и вовсе в некотором отрезке $[a + \epsilon; M]$.
Значит, последовательность $\{x_{n_k}\}$ имеет частичный предел, принадлежащий $[a + \epsilon; +\infty]$.
Но частичный предел подпоследовательности $\{x_{n_k}\}$ является частичным пределом последовательности $\{x_n\}$,
следовательно, $a$ не является верхним пределом.

(2): непосредственно следует из того, что $a$~--- частичный предел.

\paragraph{Достаточность.}
Дано: выполнение условий (1) и (2).

Доказать: $a \in \R$~--- верхний предел последовательности $\{x_n\}$, т.е. наибольший из её частичных пределов.

Доказываем в два этапа: сначала докажем, что $a$~--- частичный предел,
потом~--- что он наибольший.

Этап 1.
Положим $\varepsilon = 1/k$, $k\in\mathbb{N}$.
Тогда из условия 1) при подстановке такого $\varepsilon$ следует, что
$$
	\exists(m_k \in \N)\forall(n \geq m_k)[x_n < a + 1/k]
	.
$$
В условии 2) положим $\varepsilon = 1/k$, $m=\max\{m_k,n_{k-1}\}$ (полагая $n_0=0$):
$$
	\exists(n_k > \max\{m_k,n_{k-1}\}  )[x_{n_k} > a - 1/k]
	.
$$
Тогда подпоследовательность $x_{n_k}$ сходится к $a$,
следовательно, $a$~--- частичный предел последовательности $x_n$.

Этап 2.
Покажем, что $a$~--- наибольший из частичных пределов.
Предположим противное, т.е. существует частичный предел $b$, при этом $b>a$.
Положим $\varepsilon = (b-a)/3$.
С одной стороны, в силу условия 1) в луче $[a+\varepsilon; +\infty)$ лежит
лишь конечное число членов последовательности $\{x_n\}$.
С другой стороны, в силу определения частичного предела
в интервале $(b-\varepsilon; b+\varepsilon)$
лежит бесконечноечисло членов последовательности $\{x_n\}$.
Но в силу выбора $\varepsilon$ имеем
$(b-\varepsilon; b+\varepsilon) \subset \lbrack a+\varepsilon; +\infty)$.
Полученное противоречие завершает доказательство.


\par

Аналогично формулируется характеристическое свойство нижнего предела:

\subsubsection{Теорема.}

Для того, чтобы число $a \in \R$ было нижним пределом последовательности $\{x_n\}$, необходимо и достаточно выполнения следующих двух условий:

1)$\forall(\epsilon >0 )\exists(n_0 \in \N)\forall(n \geq n_0)[x_n > a - \epsilon]$

2)$\forall(\epsilon > 0)\forall(m   \in \N)\exists(n \geq m  )[x_n < a + \epsilon]$

\subsubsection{Замечание.}
Условие (1) означает, что количество членов последовательности, меньших $a-\epsilon$, конечно.

Условие (2) означает, что количество членов подпоследовательности, меньших $a+\epsilon$, бесконечно.



\subsection{Критерий существования предела последовательности}
\subsubsection{Теорема.}

Предел последовательности существует тогда и только тогда, когда верхний и нижний пределы это последовательности равны между собой.

В таком случае предел последовательности равен верхнему и нижнему её пределу.

Т. е.
$$
\exists\left( \lim{x_n}\in\overline\R \right) \nid \left(\varliminf x_n = \varlimsup x_n\right) 
$$

$$
\exists\left( \lim{x_n}\in\overline\R \right) \Rightarrow  \varliminf x_n = \varlimsup x_n = \lim x_n 
$$



\chapter{Вещественная функция вещественного аргумента}
\section{Предел вещественной функции вещественного аргумента}
\subsection{Определение предела функции по Коши, примеры}
\begin{opr}
	Число $b$ называется пределом функции $f(x)$ при $x$, стремящемся к $a$, если
	\begin{equation}
		\forall(\varepsilon > 0)\exists(\delta>0)\forall(x: 0<|x-a|<\delta)[|f(x)-b|<\varepsilon]
		.
	\end{equation}
	Пишут:
	\begin{equation}
		\lim_{x\to a} f(x) = b
		.
	\end{equation}
\end{opr}
Сама функция $f$ может и не быть определена в точке $a$, например
\begin{equation}
	\lim_{x\to 0} \frac{\sin x}{x} = 0
	,
\end{equation}

\begin{equation}
	\lim_{x\to 1} \frac{x^2-1}{x-1} = 2
	.
\end{equation}

Вообще говоря, предела у функции может и не существовать.
Например, не существует предела
\begin{equation}
	\lim_{x\to 0} \frac{|x|}{x}
	.
\end{equation}

\subsection{Определение предела функции по Гейне, примеры, эквивалентность определений}
\subsection{Обобщение понятия предела функции на расширенную числовую ось}
...

\section{Свойства пределов функции и функций, имеющих предел}
\subsection{Свойства, связанные с неравенствами}
\subsection{Свойства, связанные с арифметическими  операциями}
\subsection{Рецептурный подход к вычислению пределов}
Материал данного пункта относится скорее к практическим занятиям
и содержит указания о том, как вычислять пределы встречающихся на практике выражений.

\paragraph{Алгоритм действий при вычислении предела функции}
Пусть дан предел:
\begin{equation}
	\lim_{x\to a} f(x)
	,
\end{equation}
где $f(x)$ "--- <<достаточно хорошая>> функция.
Большинство встречающихся на практике функций <<достаточно хороши>>:
многочлены, степенные функции, тргонометрические функции и т.д.,
а также их сумма, разность, суперпозиция (например, $\sin (\ln x)$) и даже частное.

1. Попробуем вычислить $f(a)$.
Если это удалось "--- предел найден.
Но, как правило, это не удаётся по одной (или нескольким) из следующих причин:

а) Деление нуля на ноль. Пример:
\begin{equation}
	\lim_{x\to 1} \frac{x^2-1}{x-1}
	.
\end{equation}
В таком случае необходимо либо сокращать дробь <<вручную>> (для многочленов часто помогает схема Горнера при $x_0 = a$),
либо применять правило Лопиталя (см. ниже).


б) Деление ненулевого числа на ноль. Пример:
\begin{equation}
	\lim_{x\to 0} \frac{x^2+1}{x}
	.
\end{equation}
В таком случае возникает бесконечность. Пишут: $\infty$.

в) Деление бесконечности на бесконечность. Пример:
\begin{equation}
	\lim_{x\to \infty} \frac{x^2+1}{x^3+3}
	.
\end{equation}
Действия "--- как при делении нуля на ноль.
В нашем случае:
\begin{equation}
	\lim_{x\to \infty} \frac{x^2+1}{x^3+3}
	=
	\lim_{x\to \infty} \frac{x^{-1}+\frac1{x^3}}{1+\frac{3}{x^3}}
	=
	0
	.
\end{equation}

г) Вычитание бесконечности из бесконечности. Пример:
\begin{equation}
	\lim_{x\to \infty} \sqrt{2x-1} - \sqrt{x+1}
	.
\end{equation}
В ряде случаев помогает умножение на сопряжённое или вынос общего множителя за скобки.

д) Степенные неопределённости: $1^\infty$, $0^0$, $\infty^0$.
Пример (второй замечательный предел):
\begin{equation}
	\lim_{x \to \infty}\left(1 + \frac{1}{x}\right)^x =
	\lim_{x \to \infty}e^{x \cdot \ln\left(1 + \frac{1}{x}\right)} =
	e^{\lim_{x \to \infty} x \cdot \ln\left(1 + \frac{1}{x}\right)}
	e^{\lim_{x \to \infty} \frac{\ln\left(1 + \frac{1}{x}\right)}{1/x}}
\end{equation}
(дальнейшее может быть вычислено по правилу Лопиталя).
Обратим внимание читателя на следующие два приёма.
Во-первых, определение логарифма:
\begin{equation}
	a = e^{\ln a}, ~~\mbox{в частности,}~~  a^b = e^{b\cdot \ln a}
	.
\end{equation}
Во-вторых, функция $\exp(x)=e^x$ непрерывна и, следовательно, её можно менять местами со знаком предела:
\begin{equation}
	\lim_{x \to c} e^{f(x)} = e^{ \lim_{x \to c}f(x)}
	.
\end{equation}


\paragraph{Задачи для самостоятельного решения}
\begin{enumerate}
	\item
		\begin{equation}
			\lim_{x\to3}\frac{x^2-4}{x^2+x-6}
		\end{equation}
	\item
		\begin{equation}
			\lim_{x\to2}\frac{x^2-4}{x^2+x-6}
		\end{equation}
	\item
		\begin{equation}
			\lim_{x\to\infty}\frac{x^2+x^4}{x^3-x^2}
		\end{equation}
	\item
		\begin{equation}
			\lim_{x\to\infty}\frac{x^2+x^4}{x^5-x^2}
		\end{equation}
	\item
		\begin{equation}
	\lim_{x\to \infty} \sqrt{x-1} - \sqrt{x+1}
		\end{equation}
\end{enumerate}

...

\section{Односторонние пределы функции}
\subsection{Определение односторонних пределов, связь между существованием предела и односторонних пределов функции}
\subsection{Теорема о существовании односторонних пределов у монотонной функции и её следствия}
...

\section{Критерий Коши, замечательные пределы, бесконечно малые функции}
\subsection{Критерий Коши существования предела функции}
\subsection{Первый замечательный предел}
\subsection{Второй замечательный предел}
\subsection{Бесконечно малые функции и их классификация}
...

\section{Непрерывные функции. Общие свойства}
\subsection{Понятие непрерывности функции в точке}
\subsubsection{Определение непрерывности функиции в точке по Коши.}

\opred

\fXR, $x_0 \in X$.
Функция $f$ непрерывна в точке $x_0$, если
$$
\forall(\epsilon>0)\exists(\delta>0)[|x-x_0|<\delta \Rightarrow |f(x)-f(x_0)|<\epsilon].
$$

Или, что то же самое, но с применением окрестностей:

$$
\forall(\epsilon>0)\exists(\delta>0) [f(U_{\delta}(x_0) \cap X) \subset U_{\epsilon}(f(x_0))]
$$

Или, что то же самое:

$$
\forall(\epsilon>0)\exists(\delta>0) [f(U_{\delta,X}(x_0)) \subset U_{\epsilon}(f(x_0))]
$$

И, наконец, полностью перейдя в термины окрестностей:

$$
\forall(U \in O(f(x_0)))\exists(V \in O_X(x_0)) [f(V) \subset U]
$$

\subsubsection{Замечание 1.}

Вдумчивый читатель легко заметит, что это опреденление похоже на определение предела в точке, в котором проколотые окрестности заменены на непроколотые. Несколькими строками ниже мы рассмотрим вопросч о связи непрерывности функции, её предела и её значения в данной точке.

\subsubsection{Замечание 2.}

Если $x_0$ - изолированная точка множества $X$, то
$$
 \exists(U \in O(x_0))[U \cap X = \{x_0\}] \Rightarrow f(U)=\{f(x_0)\}],
$$
т. е. найдётся окрестность точки $x_0$, образом которой явялется единственная точка, и функция $f$ в точке $x_0$ непрерывно. Однако никаких содержательных результатов этот случай не даёт, и потому в дальнейшем мы, как правило, будем рассматривать непрерывность функции, заданной на множестве точек, лишь в предельных точках этого множества.

\subsubsection{Критерий непрерывности функции в точке.}

\fXRx.
$f$ непрерывна в $x_0$ тогда и только тогда, когда 

$$
\lim_{x\to x_0}f(x)=f(x_0)
$$

\subsubsection{Следствие 1.}

\fXRx.
$f$ непрерывна в $x_0$ тогда и только тогда, когда знак предела и знак функции коммутируют, т. е. 
$$
\lim_{x \to x_0} f(x) = f(\lim_{x \to x_0}x)
$$

\subsubsection{Следствие 2.}

\fXRx, $f$ непрерывна в $x_0$, $\Delta y = f(x_0+\Delta x)-f(x_0)$.
$\Delta x \to 0$ тогда и только тогда, когда $\Delta y \to 0$

\subsubsection{Определение непрерывности в точке по Гейне.}

\fXRx.
$f$ непрерывна в $x_0$, если 
$$
\forall(\{x_n\}:x_n \in X \cap x_n \to x_0)[f(x_n)\to f(x_0)]
$$

Обозначив $\Delta x = x_n-x_0$, $\Delta x = f(x_n)-f(x_0)$, можем сформулировать:

$$
\Delta x \to 0 \Rightarrow \Delta y \to 0
$$


\subsection{Непрерывность функции на множестве}
\opred

Функция $f:X\to \R$ называется непрерывной на $X$, если она непрерывна во всех точках $x \in X$.

\opred

Если функция $f:x \to \R$ не является непрерывной в точке $x_0 \in X$, то $x_0$ называется точкой разрыва функции $f$.

\subsubsection{Замечание 1.}

Так как все точки множества $\N$ изолированны, то любая функция $f:\N \to \R$ непрерывна.

\subsection{Понятие колебания функции на множестве и в точке. Необходимое и достаточное условие непрерывности функции в точке}
\opred

\fXR, $E \subset X$, $\alpha_E=\inf_E f(x)$, $\beta_E=\sup_E f(x)$.
Тогда разность $\alpha_E-\beta_E$ называется колебанием функции $f$ на множестве $E$:

$$
\omega(f,E)=\alpha_E-\beta_E=\sup_E f(x) - \inf_E f(x)
$$

Или, что то же самое, 

$$
\omega(f,E)=\sup_{a,b \in E}(f(a)-f(b))
$$

\subsubsection{Примеры.}

$\omega(x^2,[-2;4])=16$

$\omega(\sgn x,[0;4])=1$

$\omega(\sgn x,(0;4])=0$

$\omega(\sgn x,[-1;4])=2$

\opred

\fXRx.
Величина $\lim_{\delta \to 0+}\omega(f,U_{\delta}(x_0))$ называется колебанием функции $f$ в точке $x_0$:

$$
\omega(f,x_0)=\lim_{\delta \to 0+}\omega(f,U_{\delta}(x_0))
$$

\subsubsection{Теорема.}

Пусть $f:X\to\R$.
Функция $f$ непрерывна в точке $x_0\in X$ тогда и только тогда, когда $\omega(f,x_0)=0$.

\subsection{Односторонняя непрерывность}
\subsection{Классификация точек разрыва}
\subsection{Локальные свойства непрерывных функций}
...

\section{Функции, непрерывные на отрезке}
\subsection{Теорема Больцано-Коши и следствия из неё}
\subsubsection{Теорема.}
Пусть $f:[a;b]\to \R$ и $f$ непрерывна на $[a;b]$, при этом $f(a) \cdot f(b) <0$,
т. е. на концах отрезка $[a;b]$ непрерывная на нём функция $f$ принимает значения разного знака.
Тогда $\exists(c \in (a;b))[f(c)=0]$,
т. е. хотя бы в одной точке интервала $(a;b)$ функция обращается в нуль.

\subsubsection{Замечание.}
Теорема Больцано-Коши не только утверждает существование точки, в которой функция обращается в нуль, но и фактически даёт способ её найти - методом половинного деления отрезка. Этот факт может быть применён при нахождении корня уравнения численными методами.

\subsubsection{Следствие 1 (теорема о промежуточном значении).}
\fXR, при этом $f$ непрерывна на некотором промежутке $Y \subset X$, $\{a;b\}\subset Y$, $a<b$.
Тогда $\forall(\gamma$ между $f(a)$ и $f(b))\exists(c:c\in[a;b])[f(c)=\gamma]$.

\subsubsection{Следствие 2.}
\fXR, $X$ - промежуток и $f$ непрерывна на нём.
Тогда $f(X)$ - тоже промежуток.


 


\subsection{Первая теорема Вейерштрасса}
\subsubsection{Теорема.}

Функция, непрерывная на отрезке, ограничена на нём.

\opred

Компактом (компактным множеством) называется такое множество $X$, что
$$
\forall(\{x_n\}:x_n \in X)\exists(\{x_{n_k}\})[\{x_{n_k}\}\to x_0 \in X],
$$ 
т. е. в любой последовательности точек этого множества можно выделить подпоследовательность, сходящуюся к точке этого множества.

\subsubsection{Замечание.}
Конечный или бесконечный интервал $(a;b)$, где $\{a;b\}\subset\overline\R$, не является компактом, т. к. любая подпоследовательность любой последовательности его точек, сходящейся к $a$ или $b$, сходится к не принадлежащей интервалу точке $a$ или $b$ соответственно.

Полуинтервал также не является компактом.
Предоставляем читателю доказать это самостоятельно.

\subsubsection{Обобщение первой теоремы Вейерштрасса.}

Функция, непрерывная на компакте, ограничена на нём.

\subsubsection{Замечание.}

Функция, определённая на некомпактном множестве, может быть на нём неограничена. Пример - тождественная функция $f(x)=x$ на некомпактом множестве $(-\infty;+\infty)$.


\subsection{Вторая теорема Вейерштрасса}
\subsubsection{Теорема.}

Функция, непрерывная на компакте, достигает на нём точных верхней и нижней границ множества своих значений.

\subsubsection{Следствие.}

Пусть $f:[a;b]\to\R$ и $f$ непрерывна, $\alpha=\inf(f[a;b])$, $\beta=\sup(f[a;b])$.
Тогда $f([a;b])=[f(a);f(b)]$.




\subsection{Понятие равномерной непрерывности функции. Теорема Кантора, следствия из неё}
Согласно определению непрерывности,
$f:X \to R$ непрерывна, если 
$\forall(x_0 \in X) \forall(\epsilon >0) \exists(\delta>0)[0<|x-x_0|<\delta \Rightarrow |f(x)-f(x_0)|<\epsilon] $

В общем случае $\delta$ зависит от $\epsilon$ и $x_0$, т. е. $\delta=\delta(\epsilon,x_0)$.
Однако иногда $\delta$ зависит только от $\epsilon$ и не зависит от $x_0$, т. е. $\delta=\delta(\epsilon)$.

\opred
$f(x)$ равномерно непрерывна на $X$, если
$$ \forall(\epsilon >0) \exists(\delta>0) \forall(x_0 \in X) [0<|x-x_0|<\delta \Rightarrow |f(x)-f(x_0)|<\epsilon] $$

\subsubsection{Замечание 1.}
Если $f(x)$ равномерно непрерывна на $X$, то $f(x)$ непрерывна на $X$.
(Т.~к. квантор общности $\forall$ можно переносить вправо.)

\subsubsection{Замечание 2.}
Не всякая функция $f$, непрерывная на $X$, равномерно непрерывна на $X$.
(Например: $f(x)=x^2, f:\R \to \R$.)


\subsubsection{Теорема Кантора о равномерной непрерывности.}
\fXR, $X$ - компакт и $f$ непрерывна на $X$.
Тогда $f$ равномерно непрерывна на $X$.

\subsubsection{Следствие 1.}

Если $f:[a;b] \to \R$ непрерывна на отрезке $[a;b]$, то она равномерно непрерывна на этом отрезке.

\subsubsection{Следствие 2.}

Если $f:[a;b] \to \R$ непрерывна на отрезке $[a;b]$, то
$$
\forall(\epsilon > 0) \exists (\delta > 0) \exists(a_1, b_1 : a < a_1 < b_1 < b, b_1 - a_1<\delta)[\omega(f,[a_1,b_1]<\epsilon],
$$
или, что то же самое,
$$
\forall(\epsilon > 0) \exists (отрезок \Delta \subset [a;b])[\omega(f,\Delta)<\epsilon]
$$
т. е. найдётся подотрезок, на котором колебание функции меньше любого наперёд заданного.

\subsubsection{Замечание.}



\subsection{Свойства монотонных функций. Теорема об обратной функции}
\subsubsection{Лемма 1.}

Непрерывная функция, заданная на отрезке, инъективна в том и только том случае, когда она строго монотонна.

\subsubsection{Лемма 2.}

Пусть $X \subset \mathbb{R}$.
Любая строго монотонная функция $f:X \to Y \subset \mathbb{R}$ обладает обратной функцией $f^{-1}:Y \to X$,
причём обратная функция $f^{-1}$ имеет тот же характер монотонности на $Y$, что и функция $f$ на $X$.

\subsubsection{Лемма 3.}

Пусть $X \subset \mathbb{R}$.
Монотонная функция $f:X\to \mathbb{R}$ может иметь разрывы только первого рода.

\subsubsection{Следствие 1.}

Если $a$ - точка разрыва монотонной функции $f$, то по крайней мере один из пределов функции $f$ слева или справа от $a$ определён.

\dokvo

Если $a$ - точка разрыва, то она является предельной точкой множества $X$ и, по лемме 3, точкой разрыва первого рода.
Таким образом, точка $a$ является по крайней мере правосторонней или левосторонней предельной для множества $X$, т. е. выполнено хотя бы одно из следующих условий:
\[
f(a-0)=\lim_{x \to a-0}f(x)
\]
\[
f(a+0)=\lim_{x \to a+0}f(x)
\]
Если $a$ - двусторонняя предельная точка, то существуют и конечны оба односторонних предела.

\subsubsection{Следствие 2.}

Если $a$ - точка разрыва монотонной функции $f$, то по крайней мере в одном из неравенств $f(a-0)\leq f(a)\leq f(a+0)$ - для неубывающей $f$ или $f(a-0)\geq f(a)\geq f(a+0)$ - для невозрастающей $f$, имеет место знак строгого неравенства, т. е. $f(a-0) < f(a+0)$ - для неубывающей $f$ или $f(a-0) > f(a+0)$ - для невозрастающей $f$, и в интервале, определённым этим строгим неравенством, нет ни одного значения функции.
(Также говорят: интервал свободен от значений функции.)

\subsubsection{Следствие 3.}

Интервалы, свободные от значений монотонной функции, соответствующие разным точкам разрыва этой функции, не пересекаются.

\subsubsection{Лемма 4. Критерий непрерывности монотонной функции.}

Пусть даны отрезок $X=[a;b] \subset \mathbb{R}$ и монотонная функция $f:X \to \mathbb{R}$.
$f$ непрерывна в том и только том случае, когда $f(X)$ - отрезок $Y$ с концами $f(a)$ и $а(b)$.
($f(a) \leq f(b)$ для неубывающей $f$, $f(a) \geq f(b)$ для невозрастающей $f$).

\dokvo
\neobh

Т. к. $f$ монотонна, то все её значения лежат между $f(a)$ и $f(b)$. Т. к. $f$ непрерывна, то она принимает и все промежуточные значения. Следовательно, $f(X)$ - отрезок.

\dost

\pp, т. е. что $\exists \left(c \in [a;b]\right)$ - точка разрыва $f$.
Тогда по следствию 2 леммы 3 один из интервалов: $\left(f(c-0);f(c)\right)$ или $\left(f(c);f(c+0)\right)$ - определён и не содержит значений $f$.
С другой стороны, этот интервал содержится в $Y$, т. е. $f$ принимает не все значения из $Y$, $f(X)\neq Y$. Получили противоречие.












\subsubsection{Теорема.}
Пусть $X \subset \R, f:X \to R$ и $f$ строго монотонна.
Тогда существует обратная функция $f^{-1}:Y \to X$, где $Y=f(X)$, притом $f^{-1}$ строго монотонна на $Y$ и имеет тот же характер монотонности, что и $f$ на $X$.
Если, кроме того, $X=[a;b]$ и $f$ непрерывна на отрезке $X$, то $f([a; b])$ есть отрезок с концами $f(a)$ и $f(b)$ и $f^{-1}$ непрерывна на нём.


\subsection{Непрерывность элементарных функций}
...


\chapter{Основы дифференциального исчисления}
\section{Дифференциальное исчисление функции одной независимой переменной}
\subsection{Определение производной и дифференциала, связь между этими понятиями}
Пусть $X \subset \mathbb {R}$ $X$ - промежуток (отрезок, ...) $f : X \rightarrow \mathbb {R}$.

\opred

Производная $f(x)$ $f'(x)$ - это конечный $lim_{h \to 0} \frac {f(x_0 + h) - f(x_0)}{h}$, где $x_0 \in X$.

$\Delta f(x,h)$ - приращение функции в $x_o$.

$\Delta x = h$ - приращение аргумента.

$\Delta f(x,h) = f(x_0 + h) - f(x_0)$.

Тогда $f'(x_0)= lim_{h \to 0} \frac{\Delta f(x_0,h)}{h}$,

Если $x_0$ - фиксированное, то $f'(x_0) \in \mathbb {R}$

Если $x_0$ - нефиксированное, $a$ пробегает $X$, то $f'(x_0)$ - новая функция от $x$.

\opred
	
Функция $f(x)$ - дифференцируема в $x_0$, если существует такая линейная функция, что $\forall(h: x_0 + h \in X)[f(x_0 + h)-f(x_0) = l(x_0)h +	\omega(x_0, h)]$,

где $\omega(x_0, h)$ - бесконечно малая функция при $h \rightarrow 0$ более высокого порядка, чем $h$.

Значит $\omega(x_0, h) = 0(h)$, т.е. $lim_{h \to 0} \frac {\omega(x_0, h)}{h} = 0$.

Тогда $\Delta f(x_0, h) = l(x_0)h + \omega(x_0,h)$.

\subsubsection{Например}

$f(x)=x^3$

$\Delta f(x_0, h) = f(x_0 + h)-f(x_0) = x_0^3+3hx_0^2+3h^2x_0+h^3-x_0^3 = $

$3x_0^2h + 3h^2x_0 + h^3$.

В данном примере $f(x)=x^3$ - дифференцируема и $l(x_0) = 3x_0^2$.

\opred

Дифференциал $f(x)$ на элементе $h$ в точке $x_0$ $(df(x_0,h)) =$ значение $l(x_0)$.

Значит в нашем примере : $df = 3x_0^2$ 

\begin{teorema}

Связь между производной и дифференциалом.

Для того, чтобы $f: X \rightarrow \mathbb {R}$, где $X$ - промежуток из $\mathbb {R}$, была дифференцируема в $x_0$.

Необходимо и достаточно, чтобы $\exists (f'(x_0))$, и в уравнении 

$f(x_0 + h) - f(x_0) = l(x_0)h +	\omega(x_0, h)$. $[l(x_0=f'(x_0)]$

\end{teorema}

Доказательство:

Необходимость: дано - $f$ - дифференцируема, 

доказать $f(x_0 + h) - f(x_0) = l(x_0)h +	\omega(x_0, h)$

$\frac {f(x_0 + h) - f(x_0)}{h} = \frac {l(x_0)h}{h} + \frac {\omega(x_0, h)}{h}$;

$lim_{h \to 0}\frac {f(x_0 + h) - f(x_0)}{h} = lim_{h \to 0}\frac {l(x_0)h}{h} + lim_{h \to 0}\frac {\omega(x_0, h)}{h} = l(x_0)$;	

$lim_{h \to 0}\frac {\omega(x_0, h)}{h} = 0$ т.к. $\omega(x_0, h)$ - б.м. функция более высокого порядка, чем $h$).

Отсюда: если $\exists$ конечный $lim_{h \to 0} l(x_0)$ ($l(x_0)$ - предел от правой части равенства). Значит $\exists$ конечный $lim_{h \to 0} \frac {f(x_0 +h) - f(x_0)}{h}$.

По определению $lim_{h \to 0} \frac {f(x_0 +h) - f(x_0)}{h} = f'(x_0)$.

Отсюда $f'(x_0) = l(x_0)$.

Достаточность: Дано: $\exists (f'(x_0) = l(x_0))$. Доказать $f'(x_0)$ отличается от $lim_{h \to 0} \frac {f(x_0 +h) - f(x_0)}{h}$ на бесконечно малое число. 

Значит, если введем $\alpha(x_0, h) = \frac {f(x_0 +h) - f(x_0)}{h} - f'(x_0)$, то при $h \to 0$, $\alpha \to 0$ 

Отсюда:$f(x_0 +h) - f(x_0) - f'(x_0) = f'(x_0)h + \alpha(x_0,h)h$.

Пусть $\omega = \alpha(x_0, h)h$. Тогда $lim_{h \to 0}\frac {\omega(x_0, h)}{h} = 0$.

Т.е. $\omega(x_0, h) = 0(h)$. $\omega(x_0, h)$ - б.м. более высокого порядка малости, чем $h$.

Получается: 

$$f(x_0 +h) - f(x_0) = f'(x_o)h + \omega(x_0,h).      (1)$$

\opred

$$f(x_0 +h) - f(x_0) = l(x_o)h + \omega(x_0,h).       (2) $$

а $f'(x_0)$ при нефиксированном $x_0$ - новая функция от $x_0$, т.е. $l(x_o)$.

Значит, (1) идентична (2) и $(f(x)$ - дифференцируема. Что и требовалось доказать.

Формула дифференциала: $df(x_0, h)=f'(x_0)h$

$f(x)$ - дифференцируема если:

$\Delta f(x_0,h) = l(x_0)h + \omega(x_0, h)$, где $\omega(x_0, h) = 0(h)$/

$\Delta f(x_0,h) = df(x_0,h) + \omega(x_0, h)$.

Т.е. $f(x)$ - главная линейная часть приращения функции.

На последней формуле основано нахождение значения функций в точках окрестности точки $x$,

Если известно $f(x)$: $df(x_0, h)\approx \Delta f(x_0, h)$.

$f(x_0, h) \approx f(x_0) + df(x_0, h)$.

\subsubsection{Например}.

$\sqrt[3] x $. Пусть $x=8,1 : x_0 = 8; h = 0,1$. 

Тогда $\sqrt[3] (8,1) = f(8 + 0,1) \approx f(8) +f'(8)h = \sqrt[3] 8 + \frac {1}{3} * \frac {1}{\sqrt[3] 8^2}= 2 + \frac {1}{120}$

Точки разрыва функций $f(x)$ - точки множества $X$, в которых $f(x)$ разрывна.

Если(пусть) $y=f(x)$. Тогда $df(x_0, h) = dy = f'(x_0)\Delta x$ 

$\Delta x = (x+h)-x = h$.

Если $y=x$, $\Delta x = dx$

$$dy = f'(x_0)\Delta x = f'(x_0)dx$$

$$f'(x_0) = \frac {dy}{dx}$$
\subsection{Связь между понятиями дифференцируемости и непрерывности функций}
\begin{teorema}

Если $X \subset \mathbb {R}$ $f : X \rightarrow \mathbb {R}$, $f$ - дифференцируема в $x_0 \in X$, то $f$ - непрерывна в $x_0$.

\end{teorema}

Доказательство:

Т.к. $f(x)$ - дифференцируема, то $f(x_0 +h) - f(x_0) = f'(x_0)h + \omega(x_0,h)$,
\\
где $\omega(x_0,h) = 0(h)$.

Отсюда, если в качестве $h$ взять последовательность $h_n : x + h_n \in X$, то 
\\
$f(x_0 +h_n) - f(x_0) = f'(x_0)h_n + \omega(x_0,h_n)$.

Пусть $h_n \rightarrow 0$. Тогда $x_0 +h_n \rightarrow x_0$ и  $f(x_0 + h_n) \rightarrow f(x_0)$,
\\
т.к. $f'(x_0)h_n + \omega(x_0,h_n) \rightarrow 0$, при $n \rightarrow \infty$.

Значит по определению Гейне: о непрерывности функций $f(x)$ - непрерывна в $x_0$. Что и требовалось доказать.

\subsubsection{Замечание.}

Обратная теорема неверна. Т.е. из непрерывности не следует дифференцируемость.

\subsubsection{Например.}

$y = |x|$.
\\
$(lim_{x \to +0} \frac {\Delta y}{\Delta x} = lim_{x \to +0} \frac {\Delta x}{\Delta x} = 1) \neq (lim_{x \to -0} \frac {\Delta y}{\Delta x} = lim_{x \to -0} \frac {-\Delta x}{\Delta x} = -1)$

т.е. $y = |x|$ - не дефференцируема в точке $x=0$, хотя непрерывна в ней.

\subsection{Дифференцирование и арифметические операции}
\subsection{Теорема о производной сложной функции. Инвариантность формы первого дифференциала}
\subsection{Теорема о производной обратной функции}
\subsection{Производные основных элементарных функций. Доказательство}
\subsection{Касательная к кривой. Геометрический смысл производной и дифференциала}
\subsection{Физический смысл производной и дифференциала}
\subsection{Односторонние и бесконечные производные}
\subsection{Производные и дифференциалы высших порядков}
...

\section{Основные теоремы дифференциального исчисления}
\subsection{Понятие о локальном экстремуме функции}
\opred

\fXRx.
Точка $x_0$ называется точкой локального минимума, а значение в ней - локальным минимумом функции $f$, если
$$
\exists (U(x_0)) \forall(x \in U(x_0) \cap X)[f(x) \geq f(x_0)]
$$

\opred

\fXRx.
Точка $x_0$ называется точкой локального максимума, а значение в ней - локальным максимумом функции $f$, если
$$
\exists (U(x_0)) \forall(x \in U(x_0) \cap X)[f(x) \leq f(x_0)]
$$

\opred

\fXRx.
Точка $x_0$ называется точкой строгого локального минимума, а значение в ней - строгим локальным минимумом функции $f$, если
$$
\exists (\mathring{U}(x_0)) \forall(x \in \mathring{U}(x_0) \cap X)[f(x) > f(x_0)]
$$

\opred

\fXRx.
Точка $x_0$ называется точкой строгого локального максимума, а значение в ней - строгим локальным максимумом функции $f$, если
$$
\exists (\mathring{U}(x_0)) \forall(x \in \mathring{U}(x_0) \cap X)[f(x) < f(x_0)]
$$

\opred

Точками локального экстремума называются вместе точки локального минимума или максимума.

\opred

Локальными экстремумами называются вместе локальные минимумы или максимумы.

\opred

Точками строгого локального экстремума называются вместе точки строгого локального минимума или максимума.

\opred

Строгими локальными экстремумами называются вместе строгие локальные минимумы или максимумы.

\opred

\fXR, $x_0$ - двусторонняя предельная точка $X$.
Если $x_0$ - точка локального экстремума, то она называается точкой внутреннего локального экстремума.



\subsection{Теорема Ферма}
\subsubsection{Теорема Ферма о производной в точке локального экстремума.}

\fXR, $f$ дифференцируема в точке внутреннего локального экстремума $x_0$.
Тогда $f'(x_0)=0$.

\subsubsection{Замечание 1.}

В невнутренней точке локального экстремума производная может, вообще говоря, быть не равной нулю.
Пример: $f:[-1;1]\to \R$, невнутренний локальный максимум $x_0 = 1$, $f'(x_0)=2$.

\subsubsection{Замечание 2.}
Теорема Ферма необратима.
Пример: $f:\R\to\R$, $f(x)=x^3$, $f'(0)=0$, но $f$ не имеет локальных экстремумов.




\subsection{Теорема Ролля}
\subsubsection{Теорема.}

Если $f:[a;b]\to \R$ такова, что

1) $f$ непрерывна на $[a;b]$;

2) $f$ дифференцируема на $(a;b)$;

3) $f(a)=f(b)$,

то $\exists(c \in (a;b))[f'(c)=0]$.

\subsubsection{Замечание 1.}

Геометрическая интерпретация теоремы: пусть кривая задана функцей $y=f(x)$.
Тогда между любыми двумя точками с равными ординатами, лежащими на данной кривой, найдётся такая точка, в которой касательная к данной кривой параллельна оси абсцисс.

\subsubsection{Замечание 2.}

Условие (1) избыточно: т. к. уже требуется, чтобы $f$ была дифференцируема на $(a;b)$, достаточно потребовать непрерывности $f$ в $a$ и $b$. Остальные условия существенны.

\subsubsection{Следствие. Теорема о корнях производной.}

Между любых двух корней дифференцируемой функции лежит корень её производной.

\dokvo

Применим теорему Ролля к случаю, когда $f(a)=f(b)=0$.






\subsection{Теорема Лагранжа и следствия из нее}
\subsubsection{Теорема Лагранжа о промежуточном значении (о конечных приращениях).}

Если $f:[a;b]\to \R$ такова, что

1) $f$ непрерывна на $[a;b]$;

2) $f$ дифференцируема на $(a;b)$;

то $\exists(c \in (a;b))[f(b)-f(a)=f'(c)(b-a)]$.

\subsubsection{Замечание 1.}

Равенство $f(b)-f(a)=f'(c)(b-a)$ называют формулой Лагранжа или формулой конечных приращений.

\subsubsection{Замечание 2.}

Формулу Лагранжа можно записать и в другом виде, если положить $\theta=\frac{c-a}{b-a}$:

$$
f(b)-f(a)=f'(a+\theta(b-a))(b-a)
$$

Полагая $x=a, h=b-a$, имеем

$$
f(x+h)-f(x)=f'(x+\theta h)h
$$

\subsubsection{Следствие 1.}

Функция, имеющая на промежутке равную нулю производную, постоянная на нём.

\subsubsection{Следствие 2.}

Пусть на промежутке $X$ определены и дифференцируемы две функции $f$ и $g$, притом на концах промежутка, если они в него входят, $f$ и $g$ непрерывны.
Если $\forall(x \in X)[f'(x)=g'(x)]$, то $\forall(x \in X)[f(x)-g(x)=const]$.

\subsubsection{Следствие 3.}

Функция, имеющая на промежутке ограниченную производную, равномерно непрерывна на нём.

\subsubsection{Следствие 4.}

Пусть $f:[a;b]\to \R$, $f$ непрерывна, $f$ дифференцируема на $(x_0;x_0+h)\subset [a;b]$.
Тогда правая производная $f$ в $x_0$ непрерывна.



\subsection{Теорема Коши}
\subsubsection{Теорема Коши.}

Пусть $f:[a;b]\to \R$, $g:[a;b]\to \R$, причём:

1) $f$ и $g$ непрерывны на $[a;b]$;

2) $f$ и $g$ дифференцируемы на $(a;b)$;

3)$\nexists (x \in (a;b))[g(x)=0]$

Тогда

$$
\exists (c \in (a;b))\left[ \frac{f(b)-f(a)}{g(b)-g(a)}=\frac{f'(c)}{g'(c)}\right].
$$

\subsubsection{Замечание 1.}

Теорема Коши не является следствием из теоремы Лагранжа; наоборот, теорема Лагранжа - частный случай теоремы Коши для $g(x)=x$.

\subsubsection{Замечание 2.}

Равенство $ \frac{f(b)-f(a)}{g(b)-g(a)}=\frac{f'(c)}{g'(c)}$ называют формулой конечных приращений Коши.



\section{Формула Тейлора}
\subsection{Формула Тейлора для многочлена}
\subsection{Формула Тейлора для произвольной функции. Различные формы остаточного члена формулы Тейлора}
\subsection{Локальная формула Тейлора}
\subsection{Формула Маклорена. Разложение по формуле Маклорена некоторых элементарных функций}
\subsection{Применение формулы Тейлора}
...

\section{Правило Лопиталя}
\subsection{Неопределённость. Виды неопределённостей}
Пусть даны две непрерывные на интервале $(a; b)$ функции $f(x)$ и $g(x)$, где $\{a; b\} \subset \overline{\mathbb{R}}$. Неопределённостью типа $\left[\frac{0}{0}\right]$ в точке $a$ называется предел 
\[
\lim_{x \to a+}\frac{f(x)}{g(x)}
\]
в случае, когда
\[
\lim_{x \to a+}f(x) = \lim_{x \to a+}g(x) = 0
\]
Аналогично определяются неопределённости вида $\left[\frac{\infty}{\infty}\right]$ и в точке $b$.

Другие виды неопределённостей сводятся к этим двум. Вообще говоря, неопределённость типа $\left[\frac{\infty}{\infty}\right]$ может быть сведена к типу $\left[\frac{0}{0}\right]$. Действительно, пусть
$$\lim_{x \to a+}f(x) = \lim_{x \to a+}g(x) = \infty$$
тогда
\[
\frac{f(x)}{g(x)}=\frac{\frac{1}{g(x)}}{\frac{1}{f(x)}}
\]
Однако при раскрытии неопределённостей возникает необходимость расcматривать их отдельно.

Неопределённость-произведение сводится к неопределённостям-частным двумя способами:

$$
[0 \cdot \infty]=\lim_{x\to x_0}(f(x) \cdot g(x))=\lim_{x\to x_0}\frac{f(x)}{\frac{1}{g(x)}}=\left[\frac{0}{0}\right]
$$

$$
[0 \cdot \infty]=\lim_{x\to x_0}(f(x) \cdot g(x))=\lim_{x\to x_0}\frac{g(x)}{\frac{1}{f(x)}}=\left[\frac{\infty}{\infty}\right]
$$

Неопределённости-степени сводятся с неопределённостям-произведениям (а затем - к неопределённостям-частным) через равенство 
$$
f(x) ^ {g(x)}=e^{g(x) \cdot \ln f(x)}
$$

Заметим, что это равенство, как и сам предел, имеет смысл лишь при $f(x)>0$.
Покажем, как раскрываются неопределённости-степени:

$$
[\infty ^0]=\lim_{x\to x_0}(f(x) ^{g(x)})=\lim_{x\to x_0}e^{g(x) \cdot \ln f(x)}=e^{\lim_{x\to x_0}(g(x) \cdot \ln f(x))}=e^{[\infty \cdot 0]}
$$

$$
[0^0]=\lim_{x\to x_0}(f(x) ^{g(x)})=\lim_{x\to x_0}e^{g(x) \cdot \ln f(x)}=e^{\lim_{x\to x_0}(g(x) \cdot \ln f(x))}=e^{-[0 \cdot \infty]}
$$

$$
[1 ^\infty]=\lim_{x\to x_0}(f(x) ^{g(x)})=\lim_{x\to x_0}e^{g(x) \cdot \ln f(x)}=e^{\lim_{x\to x_0}(g(x) \cdot \ln f(x))}=e^{[0 \cdot \infty]}
$$

Наконец, рассмотри раскрытие неопределённости-разности:

$$
[\infty - \infty]=\lim_{x\to x_0}(f(x) - g(x))=\lim_{x\to x_0}\left(f(x) \cdot g(x)\left(\frac{1}{f(x)}-\frac{1}{g(x)}\right)\right)=[\infty \cdot 0]
$$

Таким образом, раскрытие неопределённостей сведено к раскрытию неопределённостей-частных.

\subsection{Теорема Лопиталя}
%Докажем теперь правило Лопиталя для неопределённостей вида $\left[\frac{\infty}{\infty}\right].$

\subsubsection{Лемма об обратном пределе.}

Пусть даны функции $f$ и $g$, такие, что $lim_{x \to x_0}f(x)=lim_{x \to x_0}g(x)=+\infty$ и 
существует предел $lim_{x \to x_0} \frac{f(x)}{g(x)}$, 
тогда существует и предел $lim_{x \to x_0} \frac{g(x)}{f(x)}$.

\dokvo

По определению бесконечно большой в точке $x_0$ функции $\exists(V=\mathring{U}_\delta(x_0))\forall(x \in V$

\subsubsection{Замечание.}

Очевидно, вынос знака "минус" из-под знака предела не составляет сложности и не влияет на применимость правила.

\subsubsection{Теорема.}

Пусть даны функции $f$ и $g$, такие, что:
1)$f$ и $g$ определены на полуинтервале $(a;b]$
2)$f$ и $g$ дифференцируемы на полуинтервале $(a;b]$

\subsection{Применение правила Лопиталя}

\section{Применение дифференциального исчисления к исследованию функции одной переменной}
\subsection{Монотонные функции}
\subsubsection{Теорема.}

\fXR.
Для того, чтобы функция $f$ была неубываюшей (невозрастающей) на $X$, необходимо и достаточно, чтобы
$\forall(x \in X) [f'(x) \geq 0] (f'(x) \leq 0])$.

\dokvo 

Докажем теорему для случая неубывающей функции. Доказательство для случая невозрастающей оставляем читателю ввиду его аналогичности.

\neobh

$f$ - неубывающая функция. Возьмём $x$ и $h \neq 0$ такие, что $x\in X, x+h \in X$.

Если $h>0$, то, так как $f$ - неубывающая, $f(x+h) \geq f(x)$.
Если $h<0$, то $f(x+h) \leq f(x)$.
Значит, 

$$
\frac{ f(x+h) - f(x) }{ h } \geq 0
$$

Переходя к пределу, имеем

\[
\lim_{h\to 0} { \frac{ f(x+h) - f(x) }{ h } } = f'(x) \geq 0
\]

\dost

$f'(x) \geq 0$. Пусть $\{x_1,x_2\} \subset X, x_1 < x_2$.
Тогда на отрезке $[x_1, x_2]$ функция $f$ дифференцируема. Применим теорему Лагранжа:

$$
\exists(c \in [ x_1, x_2 ]) [f(x_2) - f(x_1) = f'(c)(x_2 - x_1)]
$$

Но $f'(c) \geq 0$ и $x_2 - x_1 > 0$. Значит, и $f(x_2) - f(x_1) \geq 0$, т. е. функция $f$ - неубывающая.

\dokno


\subsubsection{Замечание}
\fXR, $\forall(x \in X) [f'(x) > 0] (f'(x) < 0])$.
Рассуждениями, аналогичными рассуждениями в части доказательства достаточности условия предыдущей теоремы, можно показать, что в таком случае функция $f$ -- возрастающая (убывающая).
Обратное, вообще говоря, неверно.
Например, возрастающая функция $f(x)=x^3$ имеет в точке $x=0$ нулевую производную:
$f'(x)=(x^3)'=3x^2$, $f'(0)=0$.

\subsubsection{Теорема}
\fXR, $f$ дифференцируема на $X$.
Для того, чтобы $f$ была возрастающей (убывающей), необходимо и достаточно, чтобы:

1) $\forall(x \in X) [f'(x)\geq 0]$

2) $\forall([a;b] \subset X)[f'(x)\not\equiv 0]$, т. е. чтобы ни на каком отрезке внутри $X$ $f'(x)$ не обращалась в тождественный нуль.

\dokvo 

Докажем теорему для случая возрастающей функции. Доказательство для случая убывающей оставляем читателю ввиду его аналогичности.

\neobh

$f(x)$ -- возрастающая. Тогда в силу предыдущей теоремы выполнено первое условие.
Установим, что второе условие также выполнено.
\pp, т. е. что $\exists([a;b] \subset X)\forall(x \in [a;b])[f'(x)=0]$.
Тогда $f(x)$ на $[a;b]$ постоянна, и $f(a)=f(b)$, следовательно, $f$ не является возрастающей. Получили противоречие.

\dost

Так как $f'(x) \geq 0$, то по предыдущей теореме $f$ -- неубывающая, т. е.
$\forall(x_1\in X, x_2 \in X : x_1<x_2)[f(x_2) \geq f(x_1)]$.

Докажем теперь, что $f(x_2) > f(x_1)$.
\pp, т. е. что $\exists(x_1\in X, x_2 \in X : x_1<x_2)[f(x_2) = f(x_1)]$.
Тогда $\forall(x\in [x_1; x_2])[f(x)=f(x_1)=f(x_2)]$, т. е. $\forall(x\in(x_1;x_2))[f'(x)=0]$, что противоречит второму условию теоремы.

\dokno


\subsection{Экстремумы функций}

\fXR, $f$ непрерывна на $X$. Из теоремы Ферма вытекает, что точки локального экстремума следует искать среди корней производной и точек, принадлежащих $X$, в которых не существует конечная производная (т. е. производная не определена или бесконечна).

\opred
Корни производной функции называются стационарными точками этой функции.

\opred

Стационарные точки и точки, в которых не существует конечной производной, называются критическими точками первого рода или точками, подозрительными на экстремум.

\subsubsection{Замечание}
Условие $f'(x)=0$, являясь необходимым условием внутреннего локального экстремума дифференцируемой функции, не является достаточным. Классический пример -- функция $f(x)=x^3$ в точке $x=0$ имеет нулевую производную, но не имеет экстремума.

\opred 

Говорят, что при переходе через $x_0$ производная функции $f$ меняет знак с + на -, если
$$
\exists(\delta>0)(\forall(x\in(x_0-\delta;x_0))[f'(x)>0] \cap \forall(x\in(x_0;x_0+\delta))[f'(x)<0])
$$

Определения смены знака производной с - на + и отсутствия смены знака производной аналогичны; сформулировать их оставляем читателю.

\subsubsection{Теорема о смене знака производной}

\fXR, $x_0$ - критическая точка первого рода функции $f$ и функция $f$ дифференцируема в любой внутренней точке $X$, кроме, быть может, точки $x_0$.

Если при переходе через $x_0$ производная меняет знак с + на -, то $x_0$ -- точка локального максимума $f$, если с - на +, то $x_0$ -- точка локального минимума $f$, а если смены знака нет, то в точке $x_0$ нет и экстремума.

\dokvo (Для случая смены знака с + на -; случай смены знака с - на + предоставляем читателю.)
Возьмём $\forall(x \in U_\delta(x_0))$ и рассмотрим отрезок $A$ с концами $x$ и $x_0$. По теореме Лагранжа 
$$
\exists(c\in A)[f(x)-f(x_0)=f'(c)(x-x_0)].
$$

Если $x<x_0$, то $f'(c)>0, x-x_0<0$, откуда $f(x)-f(x_0)<0$.
Если $x>x_0$, то $f'(c)<0, x-x_0>0$, откуда $f(x)-f(x_0)<0$.
Имеем:
$$
\exists(\delta>0)\forall(x\in \mathring{U}_\delta(x_0))[f(x)<f(x_0)].
$$

Это в точности определение локального максимума.

\dokvo (Для случая постоянства знака производной.)
Знак разности $f(x)-f(x_0)$ будет зависеть от знака разности $x-x_0$, т. е. положения точки $x$ слева или справа от точки $x_0$, следовательно, в $x_0$ экстремума нет.

\dokno

\subsubsection{Теорема}
\fXR и функция $f$ имеет в точке $x_0\in X$ производные до n-ого порядка включительно, причём $f'(x_0)=f''(x_0)=...=f^{(n-1)}(x_0)=0$, $f^{(n)}(x_0) \neq 0$.
Тогда:

1) Если $n$ чётно, то в точке $x_0$ функция $f$ имеет экстремум, причём если $f^{(n)}(x_0)<0$, то это максимум, а если $f^{(n)}(x_0)>0$, то минимум.

2) Если $n$ нечётно, то в $x_0$ экстремума функции $f$ нет.

\dokvo (Для случая $f^{(n)}(x_0)>0$; случай $f^{(n)}(x_0)<0$ предоставляем читателю.)

Разложим $f(x)$ по формуле Тейлора в $x_0$ с остаточным членом в форме Пеано:

$$
f(x)=f(x_0)+\frac{f'(x_0)}{1!}(x-x_0)+...+\frac{f^{(n)}(x_0)}{n!}(x-x_0)^n+o(|x-x_0|^n)
$$

Так как $f'(x_0)=f''(x_0)=...=f^{(n-1)}(x_0)$ по условию теоремы, имеем:

$$
f(x)=f(x_0)+\frac{f^{(n)}(x_0)}{n!}(x-x_0)^n+o(|x-x_0|^n)
$$


$$
f(x)-f(x_0)=\frac{f^{(n)}(x_0)}{n!}(x-x_0)^n+o(|x-x_0|^n)
$$

При $x$, достаточно близких к $x_0$,

$$
\sgn(f(x)-f(x_0))=\sgn(f^{(n)}(x_0)(x-x_0)^n).
$$

Так как $f^{(n)}(x_0)>0$, то

$$
\sgn(f(x)-f(x_0))=\sgn((x-x_0)^n).
$$

Если $n$ чётно, то $\sgn((x-x_0)^n)=1$, т. е. $f(x)-f(x_0)>0$, что означает, что $x_0$ - точка минимума.

Если $n$ нечётно, то из последнего равенства имеем

$$
\sgn(f(x)-f(x_0))=\sgn(x-x_0),
$$

т. е. в любой сколь угодно малой окрестности $x_0$ разность $(x)-f(x_0)$ меняет знак, и экстремума функции нет.

\subsubsection{Замечание}

Для того, чтобы найти наибольшее (или наименьшее) значение непрерывной функции $f:[a;b] \to \R$, нужно найти её локальные максимумы (или минимумы) и сравнить значения функции в них со значениями функции на концах отрезка.

Впрочем, иногда просто вычисляют значения функции во всех критических точках.

\subsection{Выпуклые функции}
\opred
\fXR. $f$ называется вогнутой (выпуклой вниз, вогнутой вверх) на $X$, если
$$
\forall(x_1,x_2 \in X)\forall(\alpha\in[0;1])
[f((1-\alpha)x_1+\alpha x_2) \leq (1-\alpha)f(x_1)+\alpha f(x_2)]
$$
Это определение, хотя, как мы увидим далее, весьма удобно для доказательств, может вызвать вполне объяснимое недоумение.
Поясним его геометрический смысл.

Очевидно, что $\forall(x\in[x_1;x_2])\exists(\alpha\in[0;1])[x=(1-\alpha)x_1+\alpha x_2]$, то есть любую точку отрезка $[x_1;x_2]$ можно записать в том виде, которого требует определение.

Запишем теперь уравнение прямой (хорды графика функции), проходящей через точки $(x_1,f(x_1))$ и $(x_2,f(x_2))$:
$$
\frac{y-f(x_1)}{f(x_2)-f(x_1)}=\frac{x-x_1}{x_2-x_1}
$$
Или, в явном виде:
$$
y=f(x_1)+\frac{x-x_1}{x_2-x_1}(f(x_2)-f(x_1))
$$

С учётом равенства $x=(1-\alpha)x_1+\alpha x_2$ имеем:
$$
y=f(x_1)+\frac{(1-\alpha)x_1+\alpha x_2-x_1}{x_2-x_1}(f(x_2)-f(x_1))=(1-\alpha)f(x_1)+\alpha f(x_2)
$$
(приведение подобных, раскрытие скобок и прочую арифметику оставляем читателю).
Мы получили в точности правую часть неравенства из определения.
То есть определение можно понимать так: ``Для любой точки значение функции лежит ниже хорды, стягивающей любой участок графика функции, содержащий эту точку.''.

Если добавить в определение требование строго неравенства при $\alpha\in(0;1)$, то мы получим определение функции, строго выпуклой вниз.
Аналогично формулируется определение функции, выпуклой вверх:

\opred
\fXR. $f$ называется выпуклой (выпуклой вверх, вогнутой вниз) на $X$, если
$$
\forall(x_1,x_2 \in X)\forall(\alpha\in[0;1])
[f((1-\alpha)x_1+\alpha x_2) \geq (1-\alpha)f(x_1)+\alpha f(x_2)]
$$

Аналогично же вводится строгость и даётся графическое истолкование. Два вышеизложенных определения называют определениями выпуклости функции через хорды; свяжем теперь характер выпуклости со знаком второй производной.

\begin{teorema}\label{vypukl_th_1}
Пусть функция $f$ дважды дифференцируема на $(a;b)$. Тогда для того, чтобы $f$ была выпуклой вниз/вверх, необходимо и достаточно, чтобы
$$
\forall(x\in(a;b))[f''(x)\geq 0]~/~[f''(x)\leq 0]
$$
\end{teorema}

\dokvo
Доказываем для случая выпуклости вверх; случай выпуклости вниз оставляем читателю.

\neobh
Пусть функция $f$ выпукла вверх.
\pp, т. е. что
$$
\exists(x_0\in(a;b))[f''(x_0)<0].
$$
Возьмём $\forall(h:x_0\pm h\in(a;b))$. Тогда из определения выпуклой функции при $\alpha=\frac{1}{2},~x=x_0,~x_1=x_0-h,~x_2=x_0+h$ имеем:
$$
(f(x_0+h)-f(x_0))+(f(x_0-h)-f(x_0))\geq 0
$$

Применим к каждой из этих разностей формулу Лагранжа:

\begin{multline}\label{vypukl_Lagranzh}
(f(x_0+h)-f(x_0))+(f(x_0-h)-f(x_0))=
f'(x_0+\theta_1 h)h+f'(x_0-\theta_2 h)(-h)=\\=
h^2\left(  \frac{f'(x_0+\theta_1 h)-f'(x_0)}{\theta_1 h} + \frac{f'(x_0-\theta_2 h)-f'(x_0)}{-\theta_2 h} \right)  \geq 0
\end{multline}

Напомним, что в теореме Лагранжа $\theta_1,\theta_2 \in [0;1]$.
Мы предполагали, что $f''(x_0)<0$, тогда из определения второй производной
$$
\exists(h\in(a;b))\left[
\frac{f'(x_0+\theta_1 h)-f'(x_0)}{\theta_1 h}<0
~\cap~
\frac{f'(x_0-\theta_2 h)-f'(x_0)}{-\theta_2 h}<0
\right]
$$

Получили противоречие с (\ref{vypukl_Lagranzh}).

\dost
Известно, что $\forall(x\in(a;b))[f''(x)\geq 0]$.
Возьмём $\forall(x_1,x_2 \in (a;b) : x_1 < x_2)$ и $\forall(x\in(x_1;x_2))$.
Тогда $\exists(\alpha\in[0;1])[x=(1-\alpha)x_1+\alpha x_2]$.
Применим формулу Тейлора с остаточным членом в форме Лагранжа к точкам $x_1$ и $x_2$:
$$
f(x_1)=f(x)+f'(x)(x_1-x)+\frac{f''(c_1)}{2!}(x_1-x)^2
$$
$$
f(x_2)=f(x)+f'(x)(x_2-x)+\frac{f''(c_2)}{2!}(x_2-x)^2
$$
Здесь, напомним, $c_1\in[x_1;x],~c_2\in[x;x_2]$. Умножив первое равенство на $(1-\alpha)$, а второе на $\alpha$ и сложив, имеем:
$$
(1-\alpha)f(x_1)+\alpha f(x_2)=f(x)+f'(x)(x_1+\alpha x_1 - x + \alpha x + \alpha x_2 - \alpha x)+c,
$$
где
$$
c=\frac{f''(c_1)}{2}(x_1-x)^2 (1-\alpha)+\frac{f''(c_2)}{2}(x_2-x)^2 \alpha
$$
Легко видеть, что, раз $f''(x)\geq 0$, то и $c\geq 0$.
Значит, с учётом того, что $x=(1-\alpha)x_1+\alpha x_2$, 
$$
(1-\alpha)f(x_1)+\alpha f (x_2)=f(x)+f'(x)(-\alpha x_2 + \alpha x_2)+c
$$
т. е.
$$
(1-\alpha)f(x_1)+\alpha f (x_2) \leq f(x) 
$$
Это и есть определение выпуклости.
\dokno

\begin{teorema}\label{vypukl_th_2}
Пусть $\forall(x\in(a;b))\exists(f''(x))$. Для выпуклости вниз необходимо, а в случае непрерывности $f''(x)$ и достаточно, чтобы график функции $f$ лежал не ниже касательной к графику функции $f$, проведённой в точке $(x_0;f(x_0))$ для $\forall (x_0\in(a;b))$.
\end{teorema}

\dokvo

\neobh
Запишем уравнение касательной:
$$
y_K=f(x_0)+f'(x_0)(x-x_0)
$$
Обозначив $y=f(x)$ и применив формулу Тейлора с остаточным членом в форме Лагранжа, имеем:
$$
y-y_K=f(x)-f(x_0)-f'(x_0)(x-x_0)=f''(c)\frac{(x-x_0)^2}{2}
$$
Здесь $c$ лежит между $x$ и $x_0$.
По теореме \ref{vypukl_th_1} $f''(c)\geq 0$, значит, $y\geq y_K$.

\dost
\pp, т. е. что $f$ не выпукла вниз.
Тогда по теореме \ref{vypukl_th_1} $\exists(x_0\in(a;b))[f''(x_0)<0]$.
Т. к. $f''$ непрерывна, то $$\exists(\delta>0)\forall(x\in U_\delta(x_0)[f''(x)<0]$$
Но $y-y_K=f''(c)\frac{(x-x_0)^2}{2}$, т. е. $$\forall(x\in U_\delta(x_0))[y-y_K<0],$$ т. е. график функции лежит ниже касательной.
Получили противоречие.
\dokno Случай выпуклости вниз оставляем читателю.



\subsection{Точки перегиба}
\opred
\fXR, $f$ непрерывна на $X$. Точка ${x_0 \in X}$ называется точкой перегиба функции $f$, если при переходе через $x_0$ функция $f$ меняет характер выпуклости.

\begin{teorema}\label{vypukl_th_3}
\fXR, $x_0$ - точка перегиба функции $f$ и производная $f''(x)$ непрерывна в точке $x_0$.
Тогда $f''(x_0)=0$.
\end{teorema}

\dokvo

\pp, т. е. $f''(x)\neq 0$.
НТО, положим $f''(x)>0$. Запишем формулу Тейлора для $f(x)$ с остаточным членом в форме Пеано:
$$
f(x)=f(x_0)+f'(x_0)(x-x_0)+\frac{f''(x_0)}{2}(x-x_0)^2+o(|x-x_0|^2)
$$

Зная, что ордината касательной $y_K=f(x_0)+f'(x_0)(x-x_0)$ и положив $y=f(x)$, получим

$$
y-y_K=\frac{f''(x_0)}{2}(x-x_0)^2+o(|x-x_0|^2)
$$

Но выпуклость функции определяется знаком разности $y-y_K$.
В нашем случае этот знак совпадает со знаком $\frac{f''(x_0)}{2}(x-x_0)^2+o(|x-x_0|^2)$, а в некоторой окрестности точки $x_0$ -- со знаком $\frac{f''(x_0)}{2}(x-x_0)^2$, который постоянен.
Следовательно, перемены характера выпуклости в точке $x_0$ нет.
Пришли к противоречию.

\dokno

\begin{teorema}\label{vypukl_th_4}
\fXR, $f(x)$ и $f''(x)$ непрерывны в $x_0$.
Тогда для того, чтобы $x_0$ была точкой перегиба функции $f$, необходимо и достаточно, чтобы:

1) $f''(x_0)=0$

2) $f''(x)$ меняла знак при переходе через $x_0$.
\end{teorema}

\dokvo

\neobh
Вытекает из теоремы \ref{vypukl_th_3}, определения точки перегиба и теоремы \ref{vypukl_th_1}.

\dost
Вытекает из определения точки перегиба и теоремы \ref{vypukl_th_1}.




\subsection{Асимптоты кривых}
Пусть $L$ -- кривая, заданная уравнением $y=f(x)$, $x \in X$, $y \in Y$.

\opred

Кривая $L$ имеет бесконечные ветви, если по крайней мере одно из множеств $X$ или $Y$ является неограниченным.

Рассмотрим функцию $\rho(x)=\sqrt{x^2+f^2(x)}$, $x \in X$. Для того, чтобы кривая $L$ имела бесконечные ветви, необходимо и достаточно, чтобы $\rho$ была неограниченна на $X$.

\opred

Прямая $x=x_0$ называется вертикальной асимптотой кривой $L$, заданной уравнением $y=f(x)$, если $f(x) \to \pm \infty$ при $x \to x_0 \pm$, т. е. один из односторонних пределов функции бесконечен.

Горизонтальная асимптота -- это частный случай наклонной.

\opred

Пусть $f$ задана на неограниченном промежутке $X$. Прямая $y=kx+b$ называется наклонной асимптотой кривой $y=f(x)$, если
\[
\lim_{x\to + \infty}(f(x)-kx-b)=0
\]
или
\[
\lim_{x\to - \infty}(f(x)-kx-b)=0
\]

Иногда говорят об асимптоте на бесконечности, не указывая знак. Это означает, что асимптоты на $+\infty$ и $-\infty$ совпадают.

Чтобы выяснить, имеет ли кривая асимптоты и найти $k$ и $b$, разделим равенство

$$
f(x)-kx-b=o(x)
$$

(на $\pm \infty$) на $x$. Получим

$$
k=\frac{f(x)}{x}-\frac{b}{x}-o(x)=\frac{f(x)}{x}-o(x)
$$

\[
k=\lim_{x \to \pm \infty}\frac{f(x)}{x}
\]

\[
b=\lim_{x \to \pm \infty}(f(x)-kx)
\]

Очевидно, что рассуждения верны и в обратную сторону, т. е. прямая $y=kx+b$ будет асимптотой рассматриваемой кривой.

\subsubsection{Замечание}

При $\rho(x) \to \infty$, т. е. при удалении по бесконечной ветви кривой, расстояние $d(M)$ от точки $M$ кривой с координатами $(x; f(x))$ до асимптоты стремится к нулю.

Действительно, пусть $x=x_0$ -- вертикальная асимптота. Тогда $d(M)=|x-x_0|$.
Пусть теперь $y=kx+b$ - наклонная асимптота. Опустим из точки $M$ перпендикуляр $MH$ на асимптоту и перпендикуляр $MB$ на ось $Ox$ и обозначим через $A$ точку пересечения $MB$ с асимптотой. Тогда треугольник $AMH$ - прямоугольный, и катет $MH=d(M)$ в нём меньше гипотенузы $MA$, стремящейся к нулю.

Отметим, что кривая может пересекать свою асимптоту.

\subsection{Схема исследования функции}
\newcounter{N} % для создания списков, маркированных со стилями, нужен счётчик
\begin{list}{\arabic{N}.}{\usecounter{N}}

\item Находят область определения функции.

\item Проверяют функцию на чётность, нечётность и периодичность.

\item Находят точки пересечения графика функции с осями координат, если такие точки есть.

\item Исследуют функцию на непрерывность, определяют точки разрыва и их род.

\item Исследуют поведение функции при стремлении независимой переменной $x$ к точкам разрыва и границам области определения функции, включая, если это необходимо, $\pm \infty$.

\item Находят асимптоты (вертикальные и наклонные) и точки пересечения графика функции с асимптотами.

\item Находят критические точки первого рода.

\item Находят экстремумы.

\item Определяют интервалы монотонности функции.

Предыдущие три пункта удобно осуществить с помощью первой производной, сведя результаты в таблицу, где в первой строке указываются значения аргумента $x$ - интервалы и точки, во второй -- знак производной $f'(x)$, в третьей наклонной стрелкой вверх-вправо $\nearrow$ или вниз-вправо $\searrow$ указывается характер монотонности функции.

\item С помощью второй производной определяют промежутки выпуклости и точки перегиба. Здесь снова удобно составить таблицу, аналогичную предыдущей, но второй строкой внести знак второй производной $f''(x)$, а поведение функции обозначать значками $\cap$ и $\cup$.

\end{list}





\end{document}

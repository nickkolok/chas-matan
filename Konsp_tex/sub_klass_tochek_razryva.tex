\opred
Точки разрыва функций $f(x)$ - точки множества $X$, в которых $f(x)$ разрывна.

Наряду с точками, принадлежащими $D(f)$, будем рассматривать и те точки, в которых $f(x)$ неопределена, но которые являются предельными для для области определения $f(x)$.

Например для $f(x)=\frac{1}{x}$ рассматриваем и $x=0$.

Типы разрыва:

\opred
1) Устранимый разрыв.

Пусть $x \to x_0$ , $x_0$ - точка устранимого разрыва функции $f(x)$, если 

$\exists (\lim_{x \to x_0} f(x) = b)$, $b \neq \infty$, но в $x_0$ $f(x)$ либо неопределено, либо

$f(x_0) = a, a \neq b$.

\subsubsection{Пример}
a) $f(x) = \frac{sin x}{x}$, $x_0$ точка устранимого разрыва, т.к. $\lim_{} f(x) = 1$, а 

$x=0\notin D(f)$. Функция неопределена в $x=0$, но $\exists (\lim_{} f(x) = 1)$.

Значит, этот $x=0$ мы можем "внести" в график, чуть изменив условие:

б)
\begin{equation*}
f(x) = 
 \begin{cases}
   \frac{sin x}{x} &\text{$x \neq 0$}\\
   1 &\text{$x = 0$}
 \end{cases}
\end{equation*}
Теперь

$$\lim_{x \to 0}\frac{sin x}{x} = 1 =\lim_{x \to 0} f(x)$$

2) Разрыв первого рода (конечный скачок).

\opred
$x_0$ - точка разрыва первого рода, если:
$$a=\lim_{x \to x_0 + 0} f(x) \neq \lim_{x \to x_0 - 0} f(x)$$

\subsubsection{Например}
а)
$$f(x) = sign x$$
$$\lim_{x \to + 0} f(x) = 1$$
$$\lim_{x \to - 0} f(x) = -1$$
$$1 \neq -1$$

\subsubsection{Например}
б)
$$f(x) = \frac{sin x}{|x|}$$
$$\lim_{x \to x_0 + 0} f(x) = 1$$
$$\lim_{x \to x_0 - 0} f(x) = -1$$
$\lim_{x \to x_0 \pm  0} f(x)$ - существуют, но не равны.

3) Разрыв второго рода.

\opred
$x_0$ - точка разрыва первого рода, если в $x_0$ $f(x)$ не имеет хотя бы один $\lim_{x \to x_0 \pm  0}$ или один из них $\infty$

\subsubsection{Например}
а) $f(x) = sin \frac {1}{x}$ при $x=0$ - $\lim$ нет ни +0, ни -0

б)$f(x) = 3^\frac{1}{x} - \lim_{x \to x_0 + 0} = +\infty$
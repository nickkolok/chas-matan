\begin{theorem}[О линейности разложения в степенной ряд]
	Пусть $h(x)=f(x) + k\cdot g(x)$, $\delta_f \geq \delta_g > 0$, $k, x_0 \in \mathbb R$, $c_m=a_m + k\cdot b_m$,
	\begin{equation}
		f(x) = \sum_{m=0}^\infty a_m (x-x_0)^m
		\qquad
		\mbox{ сходится на}
		(x_0 - \delta_f;x_0 + \delta_f)
		;
	\end{equation}
	\begin{equation}
		g(x) = \sum_{m=0}^\infty b_m (x-x_0)^m
		\qquad
		\mbox{ сходится на}
		(x_0 - \delta_g;x_0 + \delta_g)
		.
	\end{equation}
	Тогда
	\begin{equation}
		h(x) = \sum_{m=0}^\infty c_m (x-x_0)^m
		\qquad
		\mbox{ сходится на}
		(x_0 - \delta_g;x_0 + \delta_g)
		.
	\end{equation}
\end{theorem}

\begin{proof}
	Рассмотрим частичные суммы
	\begin{equation}
		F_n(x) = \sum_{m=0}^\infty a_m (x-x_0)^m
		\qquad
		\mbox{ сходится на}
		(x_0 - \delta_f;x_0 + \delta_f)
		;
	\end{equation}
	\begin{equation}
		G_n(x) = \sum_{m=0}^\infty b_m (x-x_0)^m
		\qquad
		\mbox{ сходится на}
		(x_0 - \delta_g;x_0 + \delta_g)
		,
	\end{equation}
	\begin{equation}
		H_n(x) = \sum_{m=0}^\infty c_m (x-x_0)^m
		\qquad
		\mbox{ сходится на}
		(x_0 - \delta_g;x_0 + \delta_g)
		.
	\end{equation}
	Легко видеть, что
	\begin{equation}
		H_n(x) = F_n(x) + k G_n(x)
		.
	\end{equation}
	Переходя к пределу по $n$ и полагая, что $x\in(x_0-\delta_g; x_0+\delta_g)$, получаем
	\begin{multline*}
		\lim_{n\to\infty} H_n(x) =
		\lim_{n\to\infty} \left(F_n(x) + k G_n(x)\right) =
		\\=
		\lim_{n\to\infty} F_n(x) + k \cdot \lim_{n\to\infty} G_n(x) =
		f(x) + k g(x) = h(x)
		.
	\end{multline*}


\end{proof}

%\begin{remark}
\paragraph{Замечание.}
	Эта теорема ничего не говорит о сходимости на концах промежутка,
	но соответствующие условия несложно получить.
%\end{remark}

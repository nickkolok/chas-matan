Материал данного пункта относится скорее к практическим занятиям
и содержит указания о том, как вычислять пределы встречающихся на практике выражений.

\paragraph{Алгоритм действий при вычислении предела функции}
Пусть дан предел:
\begin{equation}
	\lim_{x\to a} f(x)
	,
\end{equation}
где $f(x)$ "--- <<достаточно хорошая>> функция.
Большинство встречающихся на практике функций <<достаточно хороши>>:
многочлены, степенные функции, тргонометрические функции и т.д.,
а также их сумма, разность, суперпозиция (например, $\sin (\ln x)$) и даже частное.

1. Попробуем вычислить $f(a)$.
Если это удалось "--- предел найден.
Но, как правило, это не удаётся по одной (или нескольким) из следующих причин:

а) Деление нуля на ноль. Пример:
\begin{equation}
	\lim_{x\to 1} \frac{x^2-1}{x-1}
	.
\end{equation}
В таком случае необходимо либо сокращать дробь <<вручную>> (для многочленов часто помогает схема Горнера при $x_0 = a$),
либо применять правило Лопиталя (см. ниже).


б) Деление ненулевого числа на ноль. Пример:
\begin{equation}
	\lim_{x\to 0} \frac{x^2+1}{x}
	.
\end{equation}
В таком случае возникает бесконечность. Пишут: $\infty$.

в) Деление бесконечности на бесконечность. Пример:
\begin{equation}
	\lim_{x\to \infty} \frac{x^2+1}{x^3+3}
	.
\end{equation}
Действия "--- как при делении нуля на ноль.
В нашем случае:
\begin{equation}
	\lim_{x\to \infty} \frac{x^2+1}{x^3+3}
	=
	\lim_{x\to \infty} \frac{x^{-1}+\frac1{x^3}}{1+\frac{3}{x^3}}
	=
	0
	.
\end{equation}

г) Вычитание бесконечности из бесконечности. Пример:
\begin{equation}
	\lim_{x\to \infty} \sqrt{2x-1} - \sqrt{x+1}
	.
\end{equation}
В ряде случаев помогает умножение на сопряжённое или вынос общего множителя за скобки.

д) Степенные неопределённости: $1^\infty$, $0^0$, $\infty^0$.
Пример (второй замечательный предел):
\begin{equation}
	\lim_{x \to \infty}\left(1 + \frac{1}{x}\right)^x =
	\lim_{x \to \infty}e^{x \cdot \ln\left(1 + \frac{1}{x}\right)} =
	e^{\lim_{x \to \infty} x \cdot \ln\left(1 + \frac{1}{x}\right)}
	e^{\lim_{x \to \infty} \frac{\ln\left(1 + \frac{1}{x}\right)}{1/x}}
\end{equation}
(дальнейшее может быть вычислено по правилу Лопиталя).
Обратим внимание читателя на следующие два приёма.
Во-первых, определение логарифма:
\begin{equation}
	a = e^{\ln a}, ~~\mbox{в частности,}~~  a^b = e^{b\cdot \ln a}
	.
\end{equation}
Во-вторых, функция $\exp(x)=e^x$ непрерывна и, следовательно, её можно менять местами со знаком предела:
\begin{equation}
	\lim_{x \to c} e^{f(x)} = e^{ \lim_{x \to c}f(x)}
	.
\end{equation}


\paragraph{Задачи для самостоятельного решения}
\begin{enumerate}
	\item
		\begin{equation}
			\lim_{x\to3}\frac{x^2-4}{x^2+x-6}
		\end{equation}
	\item
		\begin{equation}
			\lim_{x\to2}\frac{x^2-4}{x^2+x-6}
		\end{equation}
	\item
		\begin{equation}
			\lim_{x\to\infty}\frac{x^2+x^4}{x^3-x^2}
		\end{equation}
	\item
		\begin{equation}
			\lim_{x\to\infty}\frac{x^2+x^4}{x^5-x^2}
		\end{equation}
	\item
		\begin{equation}
	\lim_{x\to \infty} \sqrt{x-1} - \sqrt{x+1}
		\end{equation}
\end{enumerate}

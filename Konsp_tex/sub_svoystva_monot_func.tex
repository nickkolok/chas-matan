\subsubsection{Лемма 1.}

Непрерывная функция, заданная на отрезке, инъективна в том и только том случае, когда она строго монотонна.

\subsubsection{Лемма 2.}

Пусть $X \subset \mathbb{R}$.
Любая строго монотонная функция $f:X \to Y \subset \mathbb{R}$ обладает обратной функцией $f^{-1}:Y \to X$,
причём обратная функция $f^{-1}$ имеет тот же характер монотонности на $Y$, что и функция $f$ на $X$.

\subsubsection{Лемма 3.}

Пусть $X \subset \mathbb{R}$.
Монотонная функция $f:X\to \mathbb{R}$ может иметь разрывы только первого рода.

\subsubsection{Следствие 1.}

Если $a$ - точка разрыва монотонной функции $f$, то по крайней мере один из пределов функции $f$ слева или справа от $a$ определён.

\dokvo

Если $a$ - точка разрыва, то она является предельной точкой множества $X$ и, по лемме 3, точкой разрыва первого рода.
Таким образом, точка $a$ является по крайней мере правосторонней или левосторонней предельной для множества $X$, т. е. выполнено хотя бы одно из следующих условий:
\[
f(a-0)=\lim_{x \to a-0}f(x)
\]
\[
f(a+0)=\lim_{x \to a+0}f(x)
\]
Если $a$ - двусторонняя предельная точка, то существуют и конечны оба односторонних предела.

\subsubsection{Следствие 2.}

Если $a$ - точка разрыва монотонной функции $f$, то по крайней мере в одном из неравенств $f(a-0)\leq f(a)\leq f(a+0)$ - для неубывающей $f$ или $f(a-0)\geq f(a)\geq f(a+0)$ - для невозрастающей $f$, имеет место знак строгого неравенства, т. е. $f(a-0) < f(a+0)$ - для неубывающей $f$ или $f(a-0) > f(a+0)$ - для невозрастающей $f$, и в интервале, определённым этим строгим неравенством, нет ни одного значения функции.
(Также говорят: интервал свободен от значений функции.)

\subsubsection{Следствие 3.}

Интервалы, свободные от значений монотонной функции, соответствующие разным точкам разрыва этой функции, не пересекаются.

\subsubsection{Лемма 4. Критерий непрерывности монотонной функции.}

Пусть даны отрезок $X=[a;b] \subset \mathbb{R}$ и монотонная функция $f:X \to \mathbb{R}$.
$f$ непрерывна в том и только том случае, когда $f(X)$ - отрезок $Y$ с концами $f(a)$ и $а(b)$.
($f(a) \leq f(b)$ для неубывающей $f$, $f(a) \geq f(b)$ для невозрастающей $f$).

\dokvo
\neobh

Т. к. $f$ монотонна, то все её значения лежат между $f(a)$ и $f(b)$. Т. к. $f$ непрерывна, то она принимает и все промежуточные значения. Следовательно, $f(X)$ - отрезок.

\dost

\pp, т. е. что $\exists \left(c \in [a;b]\right)$ - точка разрыва $f$.
Тогда по следствию 2 леммы 3 один из интервалов: $\left(f(c-0);f(c)\right)$ или $\left(f(c);f(c+0)\right)$ - определён и не содержит значений $f$.
С другой стороны, этот интервал содержится в $Y$, т. е. $f$ принимает не все значения из $Y$, $f(X)\neq Y$. Получили противоречие.












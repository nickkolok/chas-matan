Введём сначала несколько вспомогательных определений.

\opred
Разбиение $T_2$, получающееся из разбиения $T_1$ путём добавления новых точек деления, называется измельчением разбиения $T_1$.
Пишут $T_2 \supset T_1$.

Часто вместо сквозной нумерации точек измельчения используют двойную, т. е. на отрезке $[x_{j-1};x_j]$ точки нумеруются как $x_{j-1,0}, ..., x_{j-1,m}$.
Заметим, что $x_{j-1,0}=x_{j-1}$, но $x_{j-1,m}<x_j=x_{j,0}$.

\opred
Пусть даны два разбиения $T_1$ и $T_2$.
Их объединением $T=T_1 \cup T_2$ называется разбиение, составленное как из точек $T_1$, так и из точек $T_2$.

Заметим, что в таком случае $T\supset T_1$, $T\supset T_2$.

\opred
Пусть $f:[a;b]\to \R$ и $(T, \xi)$ -- некоторое разбиение отрезка $[a;b]$ на $n$ подотрезков.
Интегральной суммой функции $f$ с разбиением $T$ называется сумма произведений значений функции $f$ в выбранных точках $\xi_j$ на длины соответствующих отрезков разбиения:
$$S(f,(T,\xi))=\sum_{j=1}^n f(\xi_j)\Delta x_j$$

\opred
Функция $f$ называется интегрируемой по Риману на отрезке $[a;b]$, если
\begin{equation}\label{def_opred_integral_1}
\exists(J\in\R)\forall(\varepsilon>0)\exists(\delta>0)\forall((T,\xi):d(T)<\delta)[|S(f,(T,\xi))-J|<\varepsilon]
\end{equation}
Число $J$ в этом случае называют определённым интегралом (или интегралом Римана) функции $f$ на отрезке $[a;b]$ и пишут:
$$
J=\intl_a^b f(x) dx
$$

Здесь:

$f(x)$ -- подынтегральная функция

$f(x)dx$ -- подынтегральное выражение

$[a;b]$ -- промежуток интегрирования

$a$ -- нижний предел интегрирования

$b$ -- верхний предел интегрирования

Иногда определение (\ref{def_opred_integral_1}) пишут так:
$$
\intl_a^b f(x) dx = \lim_{d(T)\to 0} S(f,(T,\xi))
$$

Но следует иметь в виду, что запись предела здесь -- символическая, а не буквальная.
Заметим вскользь, что определение (\ref{def_opred_integral_1}) можно записать в виде, очень похожем на определение предела функции по Коши:
$$\exists(J\in\R)\forall(\varepsilon>0)\exists(\delta>0)\forall((T,\xi))[d(T)<\delta \Rightarrow |S(f,(T,\xi))-J|<\varepsilon]
$$

Тот факт, что функция $f$ является интегрируемой по Риману на отрезке $[a;b]$, сокращённо записывают так:
$$f\in R[a;b]$$


\section{Скалярные функции векторного аргумента}
\subsection{Пространство \Rn}
\subsection{Нормированное пространство \Rn}
\subsection{Последовательность в \Rn.Сходимость последовательностей. Эквивалентность покоординатной сходимости}
\subsection{Замкнутые, открытые, компактные множества в \Rn}
\subsection{Функции многих переменных. Предел. Непрерывность }
...

\section{Дифференцирование скалярных функций векторного аргумента}
\subsection{Линейные функционалы в \Rn}
\subsection{Определение дифференциала скалярной функции векторного аргумента. Связь между понятиями дифференцируемости и непрерывности}
\subsection{Простейшие свойства операции дифференцирования}
\subsection{Определение производной по направлению. Связь между понятиями дифференцируемости функции по Фреше и Гато}
\subsection{Теорема Лагранжа}
\subsection{Частные производные скалярных функций векторного аргумента. Связь между существованием частных производных и дифференцируемостью функции по Фреше и Гато}
\subsection{Теорема о дифференцируемости сложной функции и следствие из неё}
\subsection{Частные производные высших порядков}
\subsection{Дифференциалы высших порядков}
\subsection{Формула Тейлора для скалярной функции векторного аргумента}


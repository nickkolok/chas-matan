Здесь и далее будем обозначать рациональные функции (они же рациональных дроби), т. е. частное двух многочленов, буквой $R$, иногда с некоторыми индексами и диакритиками, а сами многочлены -- буквами $P$, $Q$, $S$, при этом нижний индекс, подобно курсу алгебры, отводится для указания наибольшей возможной степени многочлена. Обратим внимание на то, что некоторые термины и утверждения будут заимствоваться из курса алгебры без отдельного предупреждения.

Итак, рассмотрим вопрос об интегрировании рациональной дроби $R(x)=\frac{P(x)}{Q(x)}$.
Если эта дробь неправильная, то её легко разложить на сумму многочлена и правильной дроби, которые затем интегрировать по отдельности.
Рассмотрим вопрос об интегрировании правильной рациональной дроби ${ R(x)=\frac{P_m(x)}{Q_n(x)} }$, где $m<n$.

Как известно, любой многочлен $Q_n$ представим в виде
\begin{equation}\label{integr_rac_func_mnogoch_razlozh}
Q_n(x)=a_n(x-x_1)^{v_1}\cdot...\cdot(x-x_k)^{v_k}\cdot(x^2+p_{k+1}x+q_{k+1})^{v_{k+1}}\cdot...\cdot(x^2+p_m x+q_m)^{v_m}
,\end{equation}
где $v_1+...+v_k + 2(v_{k+1}+...+v_m)=n$.

Более того, в курсе алгебры доказывается теорема, что для рациональной дроби $R(x)=\frac{P_m(x)}{Q_n(x)}$ со знаменателем, представленным в виде (\ref{integr_rac_func_mnogoch_razlozh}), существует представление
$$
R(x)=S(x)+\sum_{j=1}^k\sum_{l=1}^{v_j}\frac{a_{j,l}}{(x-x_j)^l}+\sum_{j=k+1}^m\sum_{l=1}^{v_j}\frac{b_{j,l}x+c_{j,l}}{(x^2+p_j x + q_j)^l}
$$

При этом $a_{j,l}$, $b_{j,l}$ и $c_{j,l}$ ищутся методом неопределённых коэффициентов: выписывается разложение правильной рациональной дроби на сумму элементарных дробей, элементарные дроби приводятся к общему знаменателю, коэффициенты при одинаковых степенях переменной интегрирования приравниваются. Возникает СЛУ, в которой число уравнений равно числу неизвестных. После её решения и определяются требуемые значения $a_{j,l}$,$b_{j,l}$ и $c_{j,l}$.

Итак, разложение рациональной дроби позволяет нам сформулировать следующую (фактически, уже доказанную) теорему:

\begin{teorema}
Интеграл от любой рациональной функции выражается через рациональную функцию, логарифм и арктангенс.
\end{teorema}

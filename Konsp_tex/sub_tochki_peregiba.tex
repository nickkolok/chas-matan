\opred
\fXR, $f$ непрерывна на $X$. Точка ${x_0 \in X}$ называется точкой перегиба функции $f$, если при переходе через $x_0$ функция $f$ меняет характер выпуклости.

\begin{teorema}\label{vypukl_th_3}
\fXR, $x_0$ - точка перегиба функции $f$ и производная $f''(x)$ непрерывна в точке $x_0$.
Тогда $f''(x_0)=0$.
\end{teorema}

\dokvo

\pp, т. е. $f''(x)\neq 0$.
НТО, положим $f''(x)>0$. Запишем формулу Тейлора для $f(x)$ с остаточным членом в форме Пеано:
$$
f(x)=f(x_0)+f'(x_0)(x-x_0)+\frac{f''(x_0)}{2}(x-x_0)^2+o(|x-x_0|^2)
$$

Зная, что ордината касательной $y_K=f(x_0)+f'(x_0)(x-x_0)$ и положив $y=f(x)$, получим

$$
y-y_K=\frac{f''(x_0)}{2}(x-x_0)^2+o(|x-x_0|^2)
$$

Но выпуклость функции определяется знаком разности $y-y_K$.
В нашем случае этот знак совпадает со знаком $\frac{f''(x_0)}{2}(x-x_0)^2+o(|x-x_0|^2)$, а в некоторой окрестности точки $x_0$ -- со знаком $\frac{f''(x_0)}{2}(x-x_0)^2$, который постоянен.
Следовательно, перемены характера выпуклости в точке $x_0$ нет.
Пришли к противоречию.

\dokno

\begin{teorema}\label{vypukl_th_4}
\fXR, $f(x)$ и $f''(x)$ непрерывны в $x_0$.
Тогда для того, чтобы $x_0$ была точкой перегиба функции $f$, необходимо и достаточно, чтобы:

1) $f''(x_0)=0$

2) $f''(x)$ меняла знак при переходе через $x_0$.
\end{teorema}

\dokvo

\neobh
Вытекает из теоремы \ref{vypukl_th_3}, определения точки перегиба и теоремы \ref{vypukl_th_1}.

\dost
Вытекает из определения точки перегиба и теоремы \ref{vypukl_th_1}.




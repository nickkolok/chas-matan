\opred

Пусть даны алгебраические структуры $(G, \cdot)$ и $(G', \odot)$.
Гомоморфизмом называется отображение (функция) $f:G \to G'$,
такое, что $$\forall(a,b \in G)[f(a \cdot b)=f(a) \odot f(b)]$$

\opred
Инъективный гомоморфизм называется мономорфизмом.

\opred
Сюръективный гомоморфизм называется эпиморфизмом.

\opred
Биективный гомоморфизм называется изоморфизмом.

\opred
Гомоморфизм структуры в себя называется эндоморфизмом.

\opred
Изоморфизм структуры в себя называется автоморфизмом.

\subsubsection{Пример.}

$f:M_n(\R) \to \R$, где $f(A)=|A|$, является гомоморфизмом моноидов $(M_n(\R),\cdot)$ и $(\R, \cdot)$.

\subsubsection{Теорема.}

Суперпозиция гомоморфизмов является гомоморфизмом.

\dokvo

Пусть даны алгебраические структуры $(G,\cdot)$, $(G',\odot)$ и $(G'',\times)$
и гомоморфизмы $f:G \to G'$ и $g:G' \to G''$. Нужно доказать, что $h:G \to G''$, такое, что $h=g\circ f$,
является гомоморфизмом.
Проверим определение гомоморфизма непосредственно для $\forall(a,b \in G)$:
$$
h(a \cdot b) = (g \circ f)(a \cdot b) = g(f(a \cdot b))=g(f(a)\odot f(b)) = 
$$
$$
g(f(a)) \times g(f(b)) = h(a) \times h(b)
$$

\dokno

\subsubsection{Теорема.}

При гомоморфизме групп нейтральный элемент переходит в нейтральный.

\dokvo

Пусть $(G,\cdot)$ и $(G',\odot)$ - группы с нейтральными элементами $e$ и $e'$ соответственно,
$f:G \to G'$ - гомоморфизм.
Нужно доказать, что $f(e)=e'$.

Пусть $f(e)=x$.

Тогда $x \cdot e = x =f(e)=f(e \cdot e)=f(e)\odot f(e) = x \odot x$, т. е. 
$x \odot e' = x \odot x$, откуда по закону сокращения $x=e'$. 

\dokno

\subsubsection{Замечание.}
Требование группы в теореме существенно: существуют моноиды, для которых образом нейтрального элемента при гомоморфизме является элемент, отличный от нейтрального.
Рассмотрим множества $A=\{e'; x; a\}$ и $B=\{e; c\}$. Определим коммутативные операции на этих множествах:
$e' \cdot e' = e'$\\
$e' \cdot x  = x$\\
$e' \cdot a  = a$\\
$x  \cdot x  = x$\\
$x  \cdot a  = a$\\
$a  \cdot a  = a$\\
$e \odot e  = e$\\
$e \odot c  = c$\\
$c \odot c  = c$\\

Тогда $e'$ и $e$ - нейтральные элементы моноидов $(A,\cdot)$ и $(B,\odot)$ соответственно. Введём теперь функцию $f:B \to A$: $f(e)=x$, $f(c)=a$. Несложную проверку того, что $f$ - гомоморфизм, предоставляем читателю.

\subsubsection{Теорема.}

Отображение, обратное к изоморфизму, является изоморфизмом.

\dokvo

Пусть дан изоморфизм $f:(G,\cdot)\to(G',\odot)$.
Чтобы отображение $f^{-1}:(G',\odot)\to(G,\cdot)$ было изоморфизмом, необоходимо и достаточно (по определению), чтобы оно было биективным гомоморфизмом. Так как биективность отображения обратима, достаточно доказать, что $f^{-1}$ - гомоморфизм.

Действительно, $\forall(x,y \in G')\exists!(a \in G)\exists!(b \in G)[f(a)=x, f(b)=y]$.

Тогда
$$f(a \cdot b)= f(a) \odot f(b) \Rightarrow$$
$$f^{-1}(f(a \cdot b))= f^{-1}(f(a) \odot f(b)) \Rightarrow$$
$$a \cdot b= f^{-1}(x \odot y) \Rightarrow$$
$$f^{-1}(x) \cdot f^{-1}(y)= f^{-1}(x \odot y)$$

\dokno

\subsubsection{Теорема.}
Пусть даны группы $(G,\cdot)$, $(G',\odot)$ и гомоморфизм $f:G \to G'$.
Тогда $\forall(a \in G)[f(a^{-1})=(f(a))^{-1}]$.

\dokvo

$$
f(a) \odot f(a^{-1}) = f(a \cdot a^{-1}) = f(e) = e' = f(a) \odot (f(a))^{-1}
$$

Отсюда по правилу сокращения в группе $f(a^{-1})=(f(a))^{-1}$.
\dokno

\opred
Пусть дано отображение $f:G \to G'$.
Множество $f(G) \subset G'$ называется образом отображения $f$ и обозначается $\mathrm{Im} f$.

\opred

Пусть дан гомоморфизм групп $f:G \to G'$.
Ядром $\mathrm{Ker} f$ гомоморфизма $f$ называется множество всех элементов $G$, которые гомоморфизм $f$ переводит в нейтральный элемент $e'$ группы $G'$.

Очевидно, что ядро любого гомоморфизма групп непусто, т. к. содержит единичный элемент $e$ исходной группы $G$. 

%Докажем теперь теорему, демонстрирующую ещё одно важнейшее свойство гомоморфизмов.

%\subsubsection{Теорема о равномерности гомоморфизма.}

%Пусть дан гомоморфизм групп $f:G \to G'$. Тогда количество прообразов элемента $x \in f(G)$ не зависит от выбора $x$ и равно количеству элементов в ядре гомоморфизма, т. е. 
%$$
%\forall(x \in f(G))[|f^{-1}(x)|=|\mathrm{Ker} f|].
%$$

%\dokvo

%Для $x=e$ утверждение теоремы повторяет определение ядра гомоморфизма.
%Пусть $|\mathrm{Ker} f|=m$. Обозначим нейтральный элемент группы $G'$ через $e'$. Для произвольного $x$ имеем:
%Пусть $\mathrm{Ker} f = {}$

%$\forall(x \in f(G))\exists(a\in G)[f(a)=x]$
%$\forall(d \in \mathrm{Ker} f )[f(a \cdot d)=f(a) \odot f(d) = f(a) = x]$

%Значит, прообраз $f^{-1}(x)$ включает не менее $m$ элементов. Докажем теперь, что
%$$
%\forall(b \in f(G) : f(b)=f(a))\exists(d \in \mathrm{Ker} f)[b = a \cdot d].
%$$
%Действительно,
%$$ f(b)=f(a)\Rightarrow $$
%$$ (f(a))^{-1} \odot f(b)=(f(a))^{-1} \odot f(a)\Rightarrow $$
%$$ (f(a))^{-1} \odot f(b)=e' \Rightarrow $$
%$$ f(a^{-1}) \odot f(b)=e' \Rightarrow $$
%$$ f(a^{-1} \cdot b)=e' \Rightarrow $$

%$$ \exists(d \in \mathrm{Ker} f)[a^{-1} \cdot b=d] \Rightarrow $$
%$$ \exists(d \in \mathrm{Ker} f)[ b=d \cdot a] \Rightarrow $$
%Возьмём произвольный элемент $x\in f(G)$. Тогда найдётся элемент $a\in G$

\subsubsection{Теорема: критерий мономорфизма.}
Гомоморфизм групп является мономорфизмом тогда и только тогда, когда его ядро состоит ровно из нейтрального элемента.

\dokvo
Пусть дан гомоморфизм групп $f:(G,\cdot) \to (G',\odot)$, $e$ и $e'$ - нейтральные элементы $G$ и $G'$ соответственно.

\neobh
Пусть $f$ - мономорфизм. Тогда по определению мономорфизма $|\mathrm{Ker} f|=1$. Так как нейтральный элемент $e$ всегда входит в ядро, то $\mathrm{Ker} f=\{e\}$.

\dost

Пусть $\mathrm{Ker} f=\{e\}$ и $f(a)=f(b)$. Тогда
$$ (f(a))^{-1} \odot f(b)=(f(a))^{-1} \odot f(a)\Rightarrow $$
$$ (f(a))^{-1} \odot f(b)=e' \Rightarrow $$
$$ f(a^{-1}) \odot f(b)=e' \Rightarrow $$
$$ f(a^{-1} \cdot b)=e' \Rightarrow $$
$$ a^{-1} \cdot b = e \Rightarrow $$
$$ b = a, $$
то есть гомоморфизм $f$ инъективен, следовательно, является мономорфизмом.
\dokno




